%%%%%%%%%%%%%%%%%%%%%%%%%%%%%%%%%%%%%%%%%%%%%%%%%%%%%%%%%%%%%%%%%%%%%%%%%%%%%%%%%%%%%%%%%%%
% Yet Another Thesis Template
% LuaLaTeX Template
%
% Source:
% https://github.com/alexpovel/thesis_template
%
% Original author:
% Alex Povel -- tex(at)alexpovel.de
%
% License and Copyright:
% MIT License
%
% Copyright (c) 2019 Alex Povel
%
% Permission is hereby granted, free of charge, to any person obtaining a copy
% of this software and associated documentation files (the "Software"), to deal
% in the Software without restriction, including without limitation the rights
% to use, copy, modify, merge, publish, distribute, sublicense, and/or sell
% copies of the Software, and to permit persons to whom the Software is
% furnished to do so, subject to the following conditions:
%
% The above copyright notice and this permission notice shall be included in all
% copies or substantial portions of the Software.
%
% THE SOFTWARE IS PROVIDED "AS IS", WITHOUT WARRANTY OF ANY KIND, EXPRESS OR
% IMPLIED, INCLUDING BUT NOT LIMITED TO THE WARRANTIES OF MERCHANTABILITY,
% FITNESS FOR A PARTICULAR PURPOSE AND NONINFRINGEMENT. IN NO EVENT SHALL THE
% AUTHORS OR COPYRIGHT HOLDERS BE LIABLE FOR ANY CLAIM, DAMAGES OR OTHER
% LIABILITY, WHETHER IN AN ACTION OF CONTRACT, TORT OR OTHERWISE, ARISING FROM,
% OUT OF OR IN CONNECTION WITH THE SOFTWARE OR THE USE OR OTHER DEALINGS IN THE
% SOFTWARE.
%
% Magic Comment:
% This template requires LuaLaTeX.
% The following 'magic comment' is recognized by editors and will ensure lualatex engine use:
%!TEX TS-program = lualatex
%%%%%%%%%%%%%%%%%%%%%%%%%%%%%%%%%%%%%%%%%%%%%%%%%%%%%%%%%%%%%%%%%%%%%%%%%%%%%%%%%%%%%%%%%%%
%%%%%%%%%%%%%% The presentation preamble shares a lot of content with the regular thesis
%%%%%%%%%%%%%% preamble. However, joining both and creating a 'shared preamble' was found
%%%%%%%%%%%%%% to be pretty error-prone.
%%%%%%%%%%%%%% Using overlapping but distinct preamble files allows for two stand-alone
%%%%%%%%%%%%%% documents.
%%%%%%%%%%%%%%%%%%%%%%%%%%%%%%%%%%%%%%%%%%%%%%%%%%%%%%%%%%%%%%%%%%%%%%%%%%%%%%%%%%%%%%%%%%%
%%%%%%%%%%%%%%%%%%%%%%%%%%%%%%%%%%%%%%%%%%%%%%%%%%%%%%%%%%%%%%%%%%%%%%%%%%%%%%%%%%%%%%%%%%%
%%%%%%%%%%%%%%%%%%%%%%%%%%%%%%%%%%%%%%%%%%%%%%%%%%%%%%%%%%%%%%%%%%%%%%%%%%%%%%%%%%%%%%%%%%%
%%%%%%%%%%%%%%%%%%%%%%%%%%%%%%%%%%%%%%%%%%%%%%%%%%%%%%%%%%%%%%%%%%%%%%%%%%%%%%%%%%%%%%%%%%%
%%%%%%%%%%%%%% Typography and Misc.
%%%%%%%%%%%%%%%%%%%%%%%%%%%%%%%%%%%%%%%%%%%%%%%%%%%%%%%%%%%%%%%%%%%%%%%%%%%%%%%%%%%%%%%%%%%
%%%%%%%%%%%%%%%%%%%%%%%%%%%%%%%%%%%%%%%%%%%%%%%%%%%%%%%%%%%%%%%%%%%%%%%%%%%%%%%%%%%%%%%%%%%
%%%%%%%%%%%%%%%%%%%%%%%%%%%%%%%%%%%%%%%%%%%%%%%%%%%%%%%%%%%%%%%%%%%%%%%%%%%%%%%%%%%%%%%%%%%
%\usepackage{showframe}% Debugging
\usepackage{blindtext}% Debugging
%%%%%%%%%%%%%%%%%%%%%%%%%%%%%%%%%%%%%%%%%%%%%%%%%%%%%%%%%%%%%%%%%%%%%%%%%%%%%%%%%%%%%%%%%%%
\usepackage{amsmath}% Required before fontspec/unicode-math
\usepackage{mathtools}% AMSmath extension; coloneqq etc.
%%%%%%%%%%%%%%%%%%%%%%%%%%%%%%%%%%%%%%%%%%%%%%%%%%%%%%%%%%%%%%%%%%%%%%%%%%%%%%%%%%%%%%%%%%%
%%%%%%%%%%%%%%%%%%%%%%%%%%%%%%%%%%%%%%%%%%%%%%%%%%%%%%%%%%%%%%%%%%%%%%%%%%%%%%%%%%%%%%%%%%%
%%%%%%%%%%%%%% Theme: https://github.com/matze/mtheme
%%%%%%%%%%%%%%%%%%%%%%%%%%%%%%%%%%%%%%%%%%%%%%%%%%%%%%%%%%%%%%%%%%%%%%%%%%%%%%%%%%%%%%%%%%%
%%%%%%%%%%%%%%%%%%%%%%%%%%%%%%%%%%%%%%%%%%%%%%%%%%%%%%%%%%%%%%%%%%%%%%%%%%%%%%%%%%%%%%%%%%%
\usetheme[%
sectionpage=none,%
titleformat frame=smallcaps,%
]{metropolis}%
%%%%%%%%%%%%%%%%%%%%%%%%%%%%%%%%%%%%%%%%%%%%%%%%%%%%%%%%%%%%%%%%%%%%%%%%%%%%%%%%%%%%%%%%%%%
%%%%%%%%%%%%%%%%%%%%%%%%%%%%%%%%%%%%%%%%%%%%%%%%%%%%%%%%%%%%%%%%%%%%%%%%%%%%%%%%%%%%%%%%%%%
%%%%%%%%%%%%%% Presentation aka beamer-specific settings
%%%%%%%%%%%%%%%%%%%%%%%%%%%%%%%%%%%%%%%%%%%%%%%%%%%%%%%%%%%%%%%%%%%%%%%%%%%%%%%%%%%%%%%%%%%
%%%%%%%%%%%%%%%%%%%%%%%%%%%%%%%%%%%%%%%%%%%%%%%%%%%%%%%%%%%%%%%%%%%%%%%%%%%%%%%%%%%%%%%%%%%
% Global default overlay specification for enumerate/itemize/actionenv etc.:
\beamerdefaultoverlayspecification{<+->}%
\setbeamercovered{highly dynamic}% Make uncovering very dynamic
%%%%%%%%%%%%%%%%%%%%%%%%%%%%%%%%%%%%%%%%%%%%%%%%%%%%%%%%%%%%%%%%%%%%%%%%%%%%%%%%%%%%%%%%%%%
\usepackage{appendixnumberbeamer}% Fix totalframenumber so that appendix doesn't count
%%%%%%%%%%%%%%%%%%%%%%%%%%%%%%%%%%%%%%%%%%%%%%%%%%%%%%%%%%%%%%%%%%%%%%%%%%%%%%%%%%%%%%%%%%%
%%%%%%%%%%%%%% Outer theme sets navigational elements/header/footer/sidebar
%%%%%%%%%%%%%%%%%%%%%%%%%%%%%%%%%%%%%%%%%%%%%%%%%%%%%%%%%%%%%%%%%%%%%%%%%%%%%%%%%%%%%%%%%%%
\useoutertheme[subsection=false]{miniframes}
\setbeamercolor{section in head/foot}{fg=normal text.bg, bg=structure.fg}% Outer theme colors
%%%%%%%%%%%%%%%%%%%%%%%%%%%%%%%%%%%%%%%%%%%%%%%%%%%%%%%%%%%%%%%%%%%%%%%%%%%%%%%%%%%%%%%%%%%
%%%%%%%%%%%%%% Numbers and indentation in ToC
%%%%%%%%%%%%%%%%%%%%%%%%%%%%%%%%%%%%%%%%%%%%%%%%%%%%%%%%%%%%%%%%%%%%%%%%%%%%%%%%%%%%%%%%%%%
\setbeamertemplate{section in toc}[sections numbered]%
\setbeamertemplate{subsection in toc}%
{\leavevmode\leftskip=3.2em\rlap{\hskip-2em\inserttocsectionnumber.\inserttocsubsectionnumber}\inserttocsubsection\par}% https://tex.stackexchange.com/a/371613/120853
%%%%%%%%%%%%%%%%%%%%%%%%%%%%%%%%%%%%%%%%%%%%%%%%%%%%%%%%%%%%%%%%%%%%%%%%%%%%%%%%%%%%%%%%%%%
%%%%%%%%%%%%%% Commands to exclude certain frames from *.nav-file/navigation bar
%%%%%%%%%%%%%%%%%%%%%%%%%%%%%%%%%%%%%%%%%%%%%%%%%%%%%%%%%%%%%%%%%%%%%%%%%%%%%%%%%%%%%%%%%%%
% https://tex.stackexchange.com/a/45038/120853
\makeatletter
\let\beamer@writeslidentry@miniframeson=\beamer@writeslidentry
\def\beamer@writeslidentry@miniframesoff{%
	\expandafter\beamer@ifempty\expandafter{\beamer@framestartpage}{}% does not happen normally
	{%else
		% removed \addtocontents commands
		\clearpage\beamer@notesactions%
	}
}
\newcommand*{\miniframeson}{\let\beamer@writeslidentry=\beamer@writeslidentry@miniframeson}
\newcommand*{\miniframesoff}{\let\beamer@writeslidentry=\beamer@writeslidentry@miniframesoff}
\makeatother
%%%%%%%%%%%%%%%%%%%%%%%%%%%%%%%%%%%%%%%%%%%%%%%%%%%%%%%%%%%%%%%%%%%%%%%%%%%%%%%%%%%%%%%%%%%
%%%%%%%%%%%%%% Print an overview with current subsection highlighted at each subsection start
%%%%%%%%%%%%%%%%%%%%%%%%%%%%%%%%%%%%%%%%%%%%%%%%%%%%%%%%%%%%%%%%%%%%%%%%%%%%%%%%%%%%%%%%%%%
\AtBeginSubsection[]{% Empty brackets: do nothing for \subsection*
	{%
		\setbeamertemplate{footline}{}% Hide footer/pagenumber; can't use plain style because we want to keep headline
		\miniframesoff
		\begin{frame}<handout:0>[noframenumbering]{Agenda}
			\tableofcontents[%
			currentsection,% Highlight this
			currentsubsection,% Highlight this
			subsectionstyle=show/shaded/hide,%https://tex.stackexchange.com/a/193981/120853
			]%
		\end{frame}%
		\miniframeson
	}%
}
%%%%%%%%%%%%%%%%%%%%%%%%%%%%%%%%%%%%%%%%%%%%%%%%%%%%%%%%%%%%%%%%%%%%%%%%%%%%%%%%%%%%%%%%%%%
\AtBeginSection[]{% Empty brackets: do nothing for \section*
	{%
		\setbeamertemplate{footline}{}% Hide footer/pagenumber; can't use plain style because we want to keep headline
		\miniframesoff
		\begin{frame}<handout:0>[noframenumbering]{Agenda}
			\tableofcontents[%
			currentsection,% Highlight this
			%currentsubsection,% Highlight this
			subsectionstyle=show/shaded/hide,%https://tex.stackexchange.com/a/193981/120853
			]%
		\end{frame}%
		\miniframeson
	}%
}
%%%%%%%%%%%%%%%%%%%%%%%%%%%%%%%%%%%%%%%%%%%%%%%%%%%%%%%%%%%%%%%%%%%%%%%%%%%%%%%%%%%%%%%%%%%
%%%%%%%%%%%%%% Other settings
%%%%%%%%%%%%%%%%%%%%%%%%%%%%%%%%%%%%%%%%%%%%%%%%%%%%%%%%%%%%%%%%%%%%%%%%%%%%%%%%%%%%%%%%%%%
% Put a square as the item, immitating what we did in the thesis.
% We can't use unicode-math symbols, so just put a small rectangle (a \rule)
% Beamer+enumitem together are problematic: https://tex.stackexchange.com/a/24491/120853
\setbeamertemplate{itemize items}{\raisebox{0.35ex}{\rule{0.55ex}{0.55ex}}}%[square]% All three levels; beamer-default [square] is too large
%%%%%%%%%%%%%%%%%%%%%%%%%%%%%%%%%%%%%%%%%%%%%%%%%%%%%%%%%%%%%%%%%%%%%%%%%%%%%%%%%%%%%%%%%%%
%%%%%%%%%%%%%%%%%%%%%%%%%%%%%%%%%%%%%%%%%%%%%%%%%%%%%%%%%%%%%%%%%%%%%%%%%%%%%%%%%%%%%%%%%%%
%%%%%%%%%%%%%% Typography and Misc.
%%%%%%%%%%%%%%%%%%%%%%%%%%%%%%%%%%%%%%%%%%%%%%%%%%%%%%%%%%%%%%%%%%%%%%%%%%%%%%%%%%%%%%%%%%%
%%%%%%%%%%%%%%%%%%%%%%%%%%%%%%%%%%%%%%%%%%%%%%%%%%%%%%%%%%%%%%%%%%%%%%%%%%%%%%%%%%%%%%%%%%%
%%%%%%%%%%%%%% Thoughts on fonts:
%%%%%%%%%%%%%% metropolis theme has issues with math font (https://github.com/matze/mtheme/issues/263)
%%%%%%%%%%%%%% The current default/best solution is to use Computer Modern; doesn't look good
%%%%%%%%%%%%%% Solution: use high-quality math font. Available as FiraMath, which is in beta.
%%%%%%%%%%%%%% It works rather well, but has bugs (e.g. with \dot not centered hor., subscripts too low)
%%%%%%%%%%%%%% Alternative: use packages of matching non-Fira fonts for math (https://tex.stackexchange.com/q/403734/120853)
%%%%%%%%%%%%%% 1. 'arevmath' looks too bold. Setting the text-fonts bolder seems unacceptable, they look very nice in the original metropolis theme and should remain unchanged
%%%%%%%%%%%%%% 2. 'eulervm' is also bugged with \dot (https://tex.stackexchange.com/q/50273/120853)
%%%%%%%%%%%%%% 3. newtxsf seems okay for now (https://tex.stackexchange.com/questions/422415/fira-sans-math-companion#comment1056174_422415)
%%%%%%%%%%%%%% Author of newtxsf package recommends it to go along with FiraSans, see their doc.
%%%%%%%%%%%%%%%%%%%%%%%%%%%%%%%%%%%%%%%%%%%%%%%%%%%%%%%%%%%%%%%%%%%%%%%%%%%%%%%%%%%%%%%%%%%
%%%%%%%%%%%%%%%%%%%%%%%%%%%%%%%%%%%%%%%%%%%%%%%%%%%%%%%%%%%%%%%%%%%%%%%%%%%%%%%%%%%%%%%%%%%
%%%%%%%%%%%%%% Uncomment this block for unicode-math font specs.
%%%%%%%%%%%%%% Currently not used because of beta-status of Fira Math.
%%%%%%%%%%%%%% We don't set a roman/main font because that should never be used in presentations.
%\usefonttheme{professionalfonts}% Prevent Beamer from overriding if high-quality fonts are used
%\usepackage{unicode-math}% Requires Unicode font input
%%%%%%%%%%%%%% Be specific with font specifications so that it's clear what happens.
%%%%%%%%%%%%%% No funny business, full control, high guarantee of compatibility across machines and time.
%\defaultfontfeatures{Path=fonts/, Extension=.otf}% All fonts
%\setsansfont[% https://tex.stackexchange.com/a/62605/120853
%	Ligatures=TeX,%
%	Path=fonts/Fira/Sans/Normal/,
%	UprightFont={*-Light},% Put Light weight as actual font
%	ItalicFont={*-LightItalic},%
%	BoldFont={*-Regular},%
%	BoldItalicFont={*-Italic}]%
%	{FiraSans}%[Color=B3EB83]% Toggle color for debugging
%\setmonofont[%
%	Path=fonts/Fira/Mono/,%
%	UprightFont={*-Regular},%
%	BoldFont={*-Medium},% Mono has weight: Regular/Medium/Bold
%	]%
%	{FiraMono}%[Color=FFCE5B]% Toggle color for debugging
%\setmathfont[%
%	Path=fonts/Fira/Math/,%
%	mathrm=sym,%set math-font-commands to new versions
%	mathit=sym,%
%	mathsf=sym,%
%	mathbf=sym,%
%	mathtt=sym]%
%	{FiraMath-Light}%[Color=E85748]% Toggle color for debugging
%%%%%%%%%%%%%%%%%%%%%%%%%%%%%%%%%%%%%%%%%%%%%%%%%%%%%%%%%%%%%%%%%%%%%%%%%%%%%%%%%%%%%%%%%%%
%%%%%%%%%%%%%%%%%%%%%%%%%%%%%%%%%%%%%%%%%%%%%%%%%%%%%%%%%%%%%%%%%%%%%%%%%%%%%%%%%%%%%%%%%%%
\usepackage{newtxsf}% For math, matches Fira Sans text font
%%%%%%%%%%%%%%%%%%%%%%%%%%%%%%%%%%%%%%%%%%%%%%%%%%%%%%%%%%%%%%%%%%%%%%%%%%%%%%%%%%%%%%%%%%%
%%%%%%%%%%%%%%%%%%%%%%%%%%%%%%%%%%%%%%%%%%%%%%%%%%%%%%%%%%%%%%%%%%%%%%%%%%%%%%%%%%%%%%%%%%%
%%%%%%%%%%%%%% System Information Banner
%%%%%%%%%%%%%%%%%%%%%%%%%%%%%%%%%%%%%%%%%%%%%%%%%%%%%%%%%%%%%%%%%%%%%%%%%%%%%%%%%%%%%%%%%%%
\usepackage{hologo}
% Using fontspec, this document is incompatible with anything but XeLaTeX/LuaLaTeX.
% LuaLaTeX can grab additional memory as needed, therefore no issues with tikz and
% no need to tikzexternalize.
% Further, contour-package and XeLaTeX are strictly incompatible (LuaLaTeX works).
% Therefore, only ever use LuaLaTeX anyway.
\usepackage{xstring}
% Print only what occurs after 'This is ' and store in macro.
% luatexbanner is provided by luatex engine.
% https://tex.stackexchange.com/a/394043/120853
\StrBehind*{\luatexbanner}{This is }[\shortbanner]
% Take input and substitute strings, in this case make plain LuaTeX string into pretty logo.
% StrSubstitue has no starred variant, therefore dekotenize.
% https://tex.stackexchange.com/a/350583/120853
\StrSubstitute{\shortbanner}{\detokenize{LuaTeX}}{\hologo{LuaTeX}}[\prettybanner]

% Get contents of files
\usepackage{catchfile}
% Store textfile data in macro.
% If required, e.g. if the build process does not involve replacing the defaults
% with current values, set defaults in that file.
\CatchFileDef{\gitmetafile}{gitmeta.txt}{}
% Store result in new expandable macro, since xstring cmds are unexpandable.
% https://tex.stackexchange.com/a/476839/120853
\StrBetween[1,1]{\gitmetafile}{version=}{;}[\GitVersion]
\StrBetween[1,2]{\gitmetafile}{shorthash=}{;}[\GitShorthash]
%%%%%%%%%%%%%%%%%%%%%%%%%%%%%%%%%%%%%%%%%%%%%%%%%%%%%%%%%%%%%%%%%%%%%%%%%%%%%%%%%%%%%%%%%%%
%%%%%%%%%%%%%%%%%%%%%%%%%%%%%%%%%%%%%%%%%%%%%%%%%%%%%%%%%%%%%%%%%%%%%%%%%%%%%%%%%%%%%%%%%%%
%%%%%%%%%%%%%% Language
%%%%%%%%%%%%%%%%%%%%%%%%%%%%%%%%%%%%%%%%%%%%%%%%%%%%%%%%%%%%%%%%%%%%%%%%%%%%%%%%%%%%%%%%%%%
%%%%%%%%%%%%%%%%%%%%%%%%%%%%%%%%%%%%%%%%%%%%%%%%%%%%%%%%%%%%%%%%%%%%%%%%%%%%%%%%%%%%%%%%%%%
\usepackage{polyglossia}% Language rules. Replacement for babel in lualatex
	\setmainlanguage[variant=uk]{english}

\usepackage[style=british]{csquotes}

\usepackage[useregional]{datetime2}% Replaces 'datetime'

\usepackage{microtype}% Advanced typesetting

\usepackage{ragged2e}% For \justifying in the Abstract

\usepackage{booktabs}% For beautiful tables; absolute must
%%%%%%%%%%%%%%%%%%%%%%%%%%%%%%%%%%%%%%%%%%%%%%%%%%%%%%%%%%%%%%%%%%%%%%%%%%%%%%%%%%%%%%%%%%%
%%%%%%%%%%%%%%%%%%%%%%%%%%%%%%%%%%%%%%%%%%%%%%%%%%%%%%%%%%%%%%%%%%%%%%%%%%%%%%%%%%%%%%%%%%%
%%%%%%%%%%%%%% Auxiliary
%%%%%%%%%%%%%%%%%%%%%%%%%%%%%%%%%%%%%%%%%%%%%%%%%%%%%%%%%%%%%%%%%%%%%%%%%%%%%%%%%%%%%%%%%%%
%%%%%%%%%%%%%%%%%%%%%%%%%%%%%%%%%%%%%%%%%%%%%%%%%%%%%%%%%%%%%%%%%%%%%%%%%%%%%%%%%%%%%%%%%%%
\usepackage{scalerel}% Scale stuff relative

\newcommand*{\scaletox}[1]{\scalerel*{\includegraphics{./logos/#1.pdf}}{X}}%logos to Xheight

\newcommand*{\simplecross}{\scaletox{cross}}
%%%%%%%%%%%%%%%%%%%%%%%%%%%%%%%%%%%%%%%%%%%%%%%%%%%%%%%%%%%%%%%%%%%%%%%%%%%%%%%%%%%%%%%%%%%
\usepackage{array}
	\newcolumntype{M}[1]{>{\centering\arraybackslash}m{#1}}% Vert.+Hor. centering
%%%%%%%%%%%%%%%%%%%%%%%%%%%%%%%%%%%%%%%%%%%%%%%%%%%%%%%%%%%%%%%%%%%%%%%%%%%%%%%%%%%%%%%%%%%%
\usepackage[outline]{contour}% Incomptatible with xelatex
	\contourlength{0.15em}% default width is not 'strong' enough
	\newcommand*{\ctrw}[1]{\hypersetup{hidelinks}\contour{bg}{#1}}% bg colour
%%%%%%%%%%%%%%%%%%%%%%%%%%%%%%%%%%%%%%%%%%%%%%%%%%%%%%%%%%%%%%%%%%%%%%%%%%%%%%%%%%%%%%%%%%%
\usepackage{hyperref}% Loads URL-package internally
\hypersetup{%
pdfpagelayout=SinglePage,% Presentations aren't TwoPage etc.
unicode,%https://tex.stackexchange.com/a/66771/120853
pdfsubject = {Your Subject},%
pdfkeywords = {Your Keywords},%
}%
\usepackage[% biber is default backend
	style=numeric-comp,% Compress numeric into ranges
	autocite=superscript,% Behaviour of all \autocite commands
	sorting=none,% None: process in order of citation, so that we have counting up from 1
	sortcites=true,% Sortcites sorts multiple \cite-entries the way the sorting-key dictates
%	backref=slide,%https://tex.stackexchange.com/a/211631/120853. Do not use hyperref for this!
	url=false,% Still prints URL for @online
	doi=false,%
	isbn=false,%
	]{biblatex}%
\addbibresource{yatt.bib}% *.bib file w/ file-suffix goes here
%%%%%%%%%%%%%%%%%%%%%%%%%%%%%%%%%%%%%%%%%%%%%%%%%%%%%%%%%%%%%%%%%%%%%%%%%%%%%%%%%%%%%%%%%%%
%%%%%%%%%%%%%% Color citations; previously: https://tex.stackexchange.com/a/165016/120853
%%%%%%%%%%%%%%%%%%%%%%%%%%%%%%%%%%%%%%%%%%%%%%%%%%%%%%%%%%%%%%%%%%%%%%%%%%%%%%%%%%%%%%%%%%%
\AtEveryCite{\color{black!40}}% https://tex.stackexchange.com/a/258415/120853
%%%%%%%%%%%%%%%%%%%%%%%%%%%%%%%%%%%%%%%%%%%%%%%%%%%%%%%%%%%%%%%%%%%%%%%%%%%%%%%%%%%%%%%%%%%
%%%%%%%%%%%%%%%%%%%%%%%%%%%%%%%%%%%%%%%%%%%%%%%%%%%%%%%%%%%%%%%%%%%%%%%%%%%%%%%%%%%%%%%%%%%
%%%%%%%%%%%%%% Math, Science etc.
%%%%%%%%%%%%%%%%%%%%%%%%%%%%%%%%%%%%%%%%%%%%%%%%%%%%%%%%%%%%%%%%%%%%%%%%%%%%%%%%%%%%%%%%%%%
%%%%%%%%%%%%%%%%%%%%%%%%%%%%%%%%%%%%%%%%%%%%%%%%%%%%%%%%%%%%%%%%%%%%%%%%%%%%%%%%%%%%%%%%%%%
\usepackage{chemformula}
\usepackage{siunitx}
	\sisetup{%
		detect-weight,% Detect surrounding font's weight
		per-mode=symbol,% symbol: kg/(s), as opposed to kgs^-1 (so we take up less space)
		unit-mode=text,% Always print in text-font (so no italics)
	}%

%\usepackage{physics}% If you can, don't use it: https://tex.stackexchange.com/q/471532/120853

\DeclareMathOperator{\Mach}{M}
%%%%%%%%%%%%%%%%%%%%%%%%%%%%%%%%%%%%%%%%%%%%%%%%%%%%%%%%%%%%%%%%%%%%%%%%%%%%%%%%%%%%%%%%%%%
%%%%%%%%%%%%%% Visuals and Graphics
%%%%%%%%%%%%%%%%%%%%%%%%%%%%%%%%%%%%%%%%%%%%%%%%%%%%%%%%%%%%%%%%%%%%%%%%%%%%%%%%%%%%%%%%%%%
\usepackage{graphicx}
	\graphicspath{{images/}}%
\usepackage{xcolor}
	% colorbrewer modified RdYlBu:
	\definecolor{rdylbu1}{RGB}{215, 48, 39}% Red
	\definecolor{rdylbu2}{RGB}{252, 141, 89}%
	\definecolor{rdylbu3}{RGB}{254, 224, 144}%
	\definecolor{rdylbu4}{RGB}{224, 243, 248}%
	\definecolor{rdylbu5}{RGB}{145, 191, 219}%
	\definecolor{rdylbu6}{RGB}{69, 117, 180}% Blue

	\definecolor{mBlue}{RGB}{124, 196, 240}% Stronger blue than in thesis to match stronger orange of metropolis theme; bit hacky
%%%%%%%%%%%%%%%%%%%%%%%%%%%%%%%%%%%%%%%%%%%%%%%%%%%%%%%%%%%%%%%%%%%%%%%%%%%%%%%%%%%%%%%%%%%
%%%%%%%%%%%%%% Tikz + Plots
%%%%%%%%%%%%%%%%%%%%%%%%%%%%%%%%%%%%%%%%%%%%%%%%%%%%%%%%%%%%%%%%%%%%%%%%%%%%%%%%%%%%%%%%%%%
\usepackage{pgfplots}

\usetikzlibrary{%
positioning,%
calc,%
tikzmark,%
arrows,
backgrounds,%
shapes,% 'regular polygon'
fit}

\usepgfplotslibrary{%
groupplots,%
units,%
}%
%%%%%%%%%%%%%%%%%%%%%%%%%%%%%%%%%%%%%%%%%%%%%%%%%%%%%%%%%%%%%%%%%%%%%%%%%%%%%%%%%%%%%%%%%%%
%%%%%%%%%%%%%% Colour Cycle for plots
%%%%%%%%%%%%%%%%%%%%%%%%%%%%%%%%%%%%%%%%%%%%%%%%%%%%%%%%%%%%%%%%%%%%%%%%%%%%%%%%%%%%%%%%%%%
\pgfplotscreateplotcyclelist{mod RdYlBu3}{%modify existing colorbrewer
	{mLightBrown},% mLightBrown as defined in metropolis theme
	{mBlue},% mDarkTeal as defined in metropolis theme
	{mDarkTeal}%no comma here
}%
\pgfplotscreateplotcyclelist{mod mark list}{%modify predefined to start without mark
every mark/.append style={solid,fill=\pgfplotsmarklistfill}\\
every mark/.append style={solid,fill=\pgfplotsmarklistfill},mark=*\\
every mark/.append style={solid,fill=\pgfplotsmarklistfill},mark=square*\\
every mark/.append style={solid,fill=\pgfplotsmarklistfill},mark=triangle*\\
every mark/.append style={solid},mark=star\\
every mark/.append style={solid,fill=\pgfplotsmarklistfill},mark=diamond*\\
every mark/.append style={solid,fill=\pgfplotsmarklistfill!40},mark=otimes*\\
every mark/.append style={solid},mark=|\\
every mark/.append style={solid,fill=\pgfplotsmarklistfill},mark=pentagon*\\
}%
%%%%%%%%%%%%%%%%%%%%%%%%%%%%%%%%%%%%%%%%%%%%%%%%%%%%%%%%%%%%%%%%%%%%%%%%%%%%%%%%%%%%%%%%%%%
%%%%%%%%%%%%%% Global plot settings
%%%%%%%%%%%%%%%%%%%%%%%%%%%%%%%%%%%%%%%%%%%%%%%%%%%%%%%%%%%%%%%%%%%%%%%%%%%%%%%%%%%%%%%%%%%
\pgfplotstableset{col sep=comma}%if ALL files are comma-sep
\pgfplotsset{%
compat = newest,%
axis lines=left,%
cycle multi list = {%
	mod mark list\nextlist%just in case; likely never reached
	linestyles*\nextlist
	mod RdYlBu3%iterate through this first; if at end, start anew using next in parent
	},%
%
unit code/.code 2 args={\si{#1#2}},% Use siunitx for units library
unit markings = {slash space},
x SI prefix/micro/.style={/pgfplots/axis base prefix={axis x base 6 prefix \micro}},% https://tex.stackexchange.com/a/224574/120853
y SI prefix/micro/.style={/pgfplots/axis base prefix={axis y base 6 prefix \micro}},
z SI prefix/micro/.style={/pgfplots/axis base prefix={axis z base 6 prefix \micro}},
%
group style = {%
	xticklabels at = edge bottom,
	xlabels at = edge bottom,
	vertical sep = 3ex,
	group size = 1 by 2,
},%
%
tuftelike/.style={%https://tex.stackexchange.com/a/155210/120853
	axis line shift=10pt,% Also shifts label
	axis line style={-},
	try min ticks=3,% https://tex.stackexchange.com/a/95753/120853
	max space between ticks=50,%
},%
minimalistic/.style={%
	tuftelike,
	thick,
	axis line style = semithick,
	tick style = {semithick, fg},% fg is current beamer foreground color
	width=0.35\textwidth,
	height=0.4\textheight,
	ytick align=inside,%puts ticks inside the plot itself
	xtick align=inside,%
	title style = {font=\footnotesize},
	xtick={% Always set min, max and middle ticks; if more desired, use 'extra x ticks={}'
		\pgfkeysvalueof{/pgfplots/xmin},
		\pgfkeysvalueof{/pgfplots/xmax},
		(\pgfkeysvalueof{/pgfplots/xmax}+\pgfkeysvalueof{/pgfplots/xmin})/2
	},
	ytick={% Same as x
		\pgfkeysvalueof{/pgfplots/ymin},
		\pgfkeysvalueof{/pgfplots/ymax},
		(\pgfkeysvalueof{/pgfplots/ymax}+\pgfkeysvalueof{/pgfplots/ymin})/2
	},
	clip = false,% https://tex.stackexchange.com/a/311194/120853
	label style={font=\tiny},%https://tex.stackexchange.com/a/300673/120853
	tick label style={font=\tiny, /pgf/number format/1000 sep={\,}},%
},%
invisible/.style={opacity=0},% https://tex.stackexchange.com/a/269558/120853
visible on/.style={alt={#1{}{invisible}}},
alt/.code args={<#1>#2#3}{%
	\alt<#1>{\pgfkeysalso{#2}}{\pgfkeysalso{#3}} % \pgfkeysalso doesn't change the path
},
}%
\tikzstyle{simblock} = [%
	draw,%
	line width=1.5pt,%
	minimum size = 1.5em,%
	rounded corners,%
	fill = white,%
	]%
\tikzstyle{triangle} = [%https://tex.stackexchange.com/a/183092/120853
	regular polygon,%
	regular polygon sides=3,%
	inner sep = 0.2ex,%fit tightly
	outer sep = 0pt,
	]%
%%%%%%%%%%%%%%%%%%%%%%%%%%%%%%%%%%%%%%%%%%%%%%%%%%%%%%%%%%%%%%%%%%%%%%%%%%%%%%%%%%%%%%%%%%%
%%%%%%%%%%%%%% Functions for plotting
%%%%%%%%%%%%%%%%%%%%%%%%%%%%%%%%%%%%%%%%%%%%%%%%%%%%%%%%%%%%%%%%%%%%%%%%%%%%%%%%%%%%%%%%%%%
\tikzset{declare function={%https://tex.stackexchange.com/a/279234/120853
cpm(\x)= (\x<=500) * (0.781*(28.98641+1.853978*((\x+273.15)/1000)-9.647459*((\x+273.15)/1000)^2+16.63537*((\x+273.15)/1000)^3+0.000117/(((\x+273.15)/1000)^2))+0.2093*(31.32234-20.23531*((\x+273.15)/1000)+57.86644*((\x+273.15)/1000)^2-36.50624*((\x+273.15)/1000)^3-0.007374/((\x+273.15)/1000)^2)  +0.0093*(20.78600+2.825911e-7*((\x+273.15)/1000)-1.464191e-7*((\x+273.15)/1000)^2+1.092131e-8*((\x+273.15)/1000)^3-3.661371e-8/((\x+273.15)/1000)^2)+0.0004*(24.99735+55.18696*((\x+273.15)/1000)-33.69137*((\x+273.15)/1000)^2+7.948387*((\x+273.15)/1000)^3-0.136638/((\x+273.15)/1000)^2)) +
(\x>500) * (0.781*(19.50583+19.88705*((\x+273.15)/1000)-8.598535*((\x+273.15)/1000)^2+1.369784*((\x+273.15)/1000)^3+0.527691/(((\x+273.15)/1000)^2))+0.2093*(31.32234-20.23531*((\x+273.15)/1000)+57.86644*((\x+273.15)/1000)^2-36.50624*((\x+273.15)/1000)^3-0.007374/((\x+273.15)/1000)^2)+0.0093*(20.78600+2.825911e-7*((\x+273.15)/1000)-1.464191e-7*((\x+273.15)/1000)^2+1.092131e-8*((\x+273.15)/1000)^3-3.661371e-8/((\x+273.15)/1000)^2)+0.0004*(24.99735+55.18696*((\x+273.15)/1000)-33.69137*((\x+273.15)/1000)^2+7.948387*((\x+273.15)/1000)^3-0.136638/((\x+273.15)/1000)^2));%SHOMATE
muoxy(\x) = 20.8e-6 * (300 + 127)/(\x + 127) * (\x / 300)^(3/2);%viscosity oxygen (mu oxygen)
%
munit(\x) = 17.9e-6 * (300 + 111)/(\x + 111) * (\x / 300)^(3/2);%viscosity nitrogen (mu nitrogen)
%
muar(\x) = 22.9e-6 * (300 + 170)/(\x + 170) * (\x / 300)^(3/2);%viscosity argon (mu argon)
%
mucoo(\x) = 15e-6 * (300 + 240)/(\x + 240) * (\x / 300)^(3/2);%viscosity carbon dioxide (mu coo)
},%
}%
%%%%%%%%%%%%%%%%%%%%%%%%%%%%%%%%%%%%%%%%%%%%%%%%%%%%%%%%%%%%%%%%%%%%%%%%%%%%%%%%%%%%%%%%%%%
%%%%%%%%%%%%%%%%%%%%%%%%%%%%%%%%%%%%%%%%%%%%%%%%%%%%%%%%%%%%%%%%%%%%%%%%%%%%%%%%%%%%%%%%%%%
%%%%%%%%%%%%%% Glossaries; dont use for presentations, but nice for consistent symbols etc
%%%%%%%%%%%%%%%%%%%%%%%%%%%%%%%%%%%%%%%%%%%%%%%%%%%%%%%%%%%%%%%%%%%%%%%%%%%%%%%%%%%%%%%%%%%
%%%%%%%%%%%%%%%%%%%%%%%%%%%%%%%%%%%%%%%%%%%%%%%%%%%%%%%%%%%%%%%%%%%%%%%%%%%%%%%%%%%%%%%%%%%
\usepackage[abbreviations, symbols, record, nomain]{glossaries-extra}
% Record for bib2gls to work.
% Nomain, aka no default glossary
% Also see https://tex.stackexchange.com/q/477658/120853
%%%%%%%%%%%%%%%%%%%%%%%%%%%%%%%%%%%%%%%%%%%%%%%%%%%%%%%%%%%%%%%%%%%%%%%%%%%%%%%%%%%%%%%%%%%
%%%%%%%%%%%%%% New Glossary 'types'. Load data into them from *.bib-files, then run bib2gls
%%%%%%%%%%%%%%%%%%%%%%%%%%%%%%%%%%%%%%%%%%%%%%%%%%%%%%%%%%%%%%%%%%%%%%%%%%%%%%%%%%%%%%%%%%%
% Using Abbreviations/Symbols package options creates these glossaries already.
% Create the rest manually.
\newglossary*{subsuper}{Sub- and Superscripts}%
\GlsXtrLoadResources[%
	src={./glossaries/abbreviations},%
	selection = {all},% Enable for debugging
]%
\GlsXtrLoadResources[%
	src={./glossaries/symbols},%
	selection = {all},% Enable for debugging
]%
\GlsXtrLoadResources[%
	src={./glossaries/subsuper},%
	type=subsuper,%
	category = subsuper,
	selection = {all},% Enable for debugging
]%
% Disable hyperlinking in presentation:
\glsdisablehyper
%%%%%%%%%%%%%%%%%%%%%%%%%%%%%%%%%%%%%%%%%%%%%%%%%%%%%%%%%%%%%%%%%%%%%%%%%%%%%%%%%%%%%%%%%%%
%%%%%%%%%%%%%%%%%%%%%%%%%%%%%%%%%%%%%%%%%%%%%%%%%%%%%%%%%%%%%%%%%%%%%%%%%%%%%%%%%%%%%%%%%%%
%%%%%%%%%%%%%% Custom Commands
%%%%%%%%%%%%%%%%%%%%%%%%%%%%%%%%%%%%%%%%%%%%%%%%%%%%%%%%%%%%%%%%%%%%%%%%%%%%%%%%%%%%%%%%%%%
%%%%%%%%%%%%%%%%%%%%%%%%%%%%%%%%%%%%%%%%%%%%%%%%%%%%%%%%%%%%%%%%%%%%%%%%%%%%%%%%%%%%%%%%%%%
% In the original thesis document, \name is used alongside \sindex to produce index entries for all names; in order to use it here, too, define it but remove index capabilities:
\newcommand*{\name}[1]{\textsc{#1}}% Just a wrapper for people's names
%%%%%%%%%%%%%%%%%%%%%%%%%%%%%%%%%%%%%%%%%%%%%%%%%%%%%%%%%%%%%%%%%%%%%%%%%%%%%%%%%%%%%%%%%%%
\newcommand*{\whitebox}[1]{\setlength{\fboxsep}{3pt}\colorbox{bg}{#1}} %https://tex.stackexchange.com/a/23682/120853
%%%%%%%%%%%%%%%%%%%%%%%%%%%%%%%%%%%%%%%%%%%%%%%%%%%%%%%%%%%%%%%%%%%%%%%%%%%%%%%%%%%%%%%%%%%
\newcommand*{\angstanitz}[1]{\ensuremath{\gls{angabs}_{#1}^{\ast}}}%Stanitz shifted by 90deg
%%%%%%%%%%%%%%%%%%%%%%%%%%%%%%%%%%%%%%%%%%%%%%%%%%%%%%%%%%%%%%%%%%%%%%%%%%%%%%%%%%%%%%%%%%%
\newcommand*{\const}{const.}%
%%%%%%%%%%%%%%%%%%%%%%%%%%%%%%%%%%%%%%%%%%%%%%%%%%%%%%%%%%%%%%%%%%%%%%%%%%%%%%%%%%%%%%%%%%%
%%%%%%%%%%%%%% Tikz
%%%%%%%%%%%%%%%%%%%%%%%%%%%%%%%%%%%%%%%%%%%%%%%%%%%%%%%%%%%%%%%%%%%%%%%%%%%%%%%%%%%%%%%%%%%
\newcommand*{\circnum}[1]{% A circled (preferably) number
%	\tikzset{external/export=false}% Never externalize this
	\tikz[baseline=(X.base)]%
	\node (X) [draw, shape=circle, inner sep=1pt, fill=bg] {\textbf{#1}};%
}%
\newcommand*{\circlet}[1]{% A circled letter
%	\tikzset{external/export=false}% Never externalize
	\tikz[baseline=(char.base)]{%
	\node [draw, shape=circle, inner sep=1pt, fill=bg] (char) {\textbf{#1}};}%
}%
%%%%%%%%%%%%%%%%%%%%%%%%%%%%%%%%%%%%%%%%%%%%%%%%%%%%%%%%%%%%%%%%%%%%%%%%%%%%%%%%%%%%%%%%%%%
%%%%%%%%%%%%%% Vectors
%%%%%%%%%%%%%%%%%%%%%%%%%%%%%%%%%%%%%%%%%%%%%%%%%%%%%%%%%%%%%%%%%%%%%%%%%%%%%%%%%%%%%%%%%%%
% The bib2gls library (*.bib-files) uses \symbf etc. from unicode-math, since
% the thesis documents is based on that package. The presentation here is not,
% therefore we provide the command manually, so the bib-files are usable.
\providecommand*{\symbf}{\mathbf}
\newcommand*{\vect}[2]{\ensuremath{\vec{\symbf{#1}}_{\text{#2}}}}% Vector
\newcommand*{\vectcomp}[3]{\ensuremath{\symbf{#1}_{\text{#2#3}}}}% Vector component
%%%%%%%%%%%%%%%%%%%%%%%%%%%%%%%%%%%%%%%%%%%%%%%%%%%%%%%%%%%%%%%%%%%%%%%%%%%%%%%%%%%%%%%%%%%
%%%%%%%%%%%%%% Physics-package-like commands
%%%%%%%%%%%%%%%%%%%%%%%%%%%%%%%%%%%%%%%%%%%%%%%%%%%%%%%%%%%%%%%%%%%%%%%%%%%%%%%%%%%%%%%%%%%
% Physics package has problems. E.g., \dd prints roman -> poor for sans-serif presentation
%\providecommand{\var}[1]{\ensuremath{\delta #1}}%
\providecommand{\dd}[1]{\ensuremath{\textnormal{d}#1}}%
%%%%%%%%%%%%%%%%%%%%%%%%%%%%%%%%%%%%%%%%%%%%%%%%%%%%%%%%%%%%%%%%%%%%%%%%%%%%%%%%%%%%%%%%%%%
%%%%%%%%%%%%%% Programs/Languages
%%%%%%%%%%%%%%%%%%%%%%%%%%%%%%%%%%%%%%%%%%%%%%%%%%%%%%%%%%%%%%%%%%%%%%%%%%%%%%%%%%%%%%%%%%%
\newcommand*{\smlnk}{%https://tex.stackexchange.com/a/471263/120853
	Simulink%
	\nobreak
	\textsuperscript{\textregistered}%
}%
\newcommand*{\mtlb}{%
	MATLAB%
	\nobreak
	\textsuperscript{\textregistered}%
}%
\newcommand*{\mtlbsymb}{%
	Symbolic Math Toolbox%
	\nobreak
	\textsuperscript{\texttrademark}%
}%
\newcommand*{\mtlbsmlnk}{%
	\mtlb % MATLAB
	\kern-0.25em % move the slash back
	\slash % allow a break after the slash
	\nobreak % the next space is not allowed for a break
	\hspace{0pt}% allow the next word to be hyphenated
	\smlnk
}%
\newcommand*{\smscp}{%
	Simscape%
	\nobreak%
	\textsuperscript{\texttrademark}%
}%
\newcommand*{\fortran}{%
FORTRAN\term{FORTRAN}%
}%