\chapter{Technical Aspects}
\TeX{} being \TeX{}, there are countless of technical aspects to pay attention to.
The most important one are collected here, in a random fashion.

\section{Building the document}
The most important part!
In \cref{ch:text}, we looked into why we are using \hologo{LuaLaTeX}.
Reasons are:
\begin{enumerate}
	\item \package{unicode-math} aka \package{fontspec}
	\item \package{contour} works (it does not with \hologo{XeLaTeX})
	\item it allocates as much memory as it happens to need.
	The ancient limitations of \TeX{} are easily reached with \package{tikz} and \package{pgfplots} functionalities.
	Circumventing that either feels hacky or actually is hacky.
	\texttt{tikzexternalize} is a decent module, but has a long list of caveats.
	Using \hologo{LuaLaTeX}, we do not have to worry about any of that: hit compile, wait, done (the waiting part might be a bit longer than what you might be used to from \hologo{pdfLaTeX}).
\end{enumerate}

With that out of the way, the building process --- starting from scratch --- looks roughly like this:
\begin{enumerate}
	\item \textbf{\hologo{LuaLaTeX}} --- build first version, generating the auxiliary files.
	At this point, references or anything else will not yet be correct.
	\item \textbf{biber} --- it will take the specified \texttt{*.bib} file and build the bibliography auxiliary files.
	\item \textbf{splitindex} --- invoke it as \texttt{splitindex NAME}, where NAME is your main file's name, such that all the auxiliary files bear names like \texttt{name.aux} \iecfeg{etc.}
	\item \textbf{bib2gls} --- also invoked as  \texttt{bib2gls NAME},
	\item \textbf{\hologo{LuaLaTeX}},
	\item \textbf{\hologo{LuaLaTeX}}.
\end{enumerate}
This list is no guarantee however.
If anything fails, try running it again in different orders, or run \hologo{LuaLaTeX} again.
You can also look into deleting auxiliary files (they can always be regenerated freely) if you are seeing really cryptic error messages.

\subsection{latexmk}
A much more user-friendly tool is \texttt{latexmk}, also provided in most \TeX{} distributions.
Therefore, you should already have it available and on your PATH.

The tool knows about all the different auxiliary files and will run your compiler of choice as many times as needed.
It will know when to stop by checking the auxiliary files: if they do not change anymore between runs, we must have hit a \enquote{steady state} and the produced document will be ready.

Since it does not natively do well with either \package{glossaries-extra} or \package{splitidx}, this project contains a file \texttt{latexmkrc} in its root.
It is a configuration file written in Perl, letting \texttt{latexmk} know about the auxiliary files of those two packages and what to do with them.
The tool will look for its configuration file in the project root and find it there.

Now, you can start from absolute scratch, with no auxiliary files generated yet at all.
All you need is a \TeX{} distribution and Java (for \package{bib2gls})!
I recommend TeXLive: in its full installation, it ships with all the required tools.

Then invoking \textbf{\texttt{latexmk yourfile}} will run the entire pipeline, giving you the fully-done end product.

\subsection{Version Control}
\LaTeX{} is well suited for a version control system, specifically Git.
Using GitHub or GitLab, you can also have your document built automatically, on servers specified (and perhaps paid for) by you.
You could be editing \LaTeX{} source code on your phone, push it to the remote repository, and download the compiled document.

If using Git, each text line should also be on its own line of code.
Should you bunch multiple text sentences into one line of code, diffing will be harder.

\subsubsection{Continuous Integration}

If you are on GitLab, you can configure your runner with the YAML file found in the project root (the runner will expect it there anyway).
The runner you need to provide yourself; perhaps, your university has one.
I do not yet have an analogous Travis-\glsxtrshort{cont_int} configuration for GitHub.

\paragraph{Docker}
The \gls{cont_int} configuration for GitLab relies on a docker image found here:
\begin{center}
	\url{https://cloud.docker.com/u/alexpovel/repository/docker/alexpovel/javalatex}
\end{center}
Its \texttt{Dockerfile} does \texttt{apt-get update -y texlive-full} and as such, will download a \emph{massive}, full TeXLive distribution.
That is still worth it and nice to have, so we do not have to specify packages manually ourselves.
The file's second important part is of course Java for \package{bib2gls}.

\paragraph{Metadata}
In the project root, you will notice a file \texttt{gitmeta.txt}.
This is my pathetic approach of breathing some dynamic metadata printing into the produced document.
You can see the result of that on \cpageref{metadata} of this document, right after the title page.
\LaTeX{} is made to read the contents of that \texttt{gitmeta.txt} file and implements it.
The text file's default contents are printed if we do not involve it in our building process.

We could change the \texttt{*.tex} contents directly using the \texttt{sed} commands from the \texttt{.gitlab-ci.yml} file, but I reckoned going for an external file would make the \texttt{*.tex} file more robust and stand-alone.

You can will the file out manually (highly discouraged and besides the point) or have it, as is configured for GitLab in \texttt{.gitlab-ci.yml}, done by the building process on your runner.

That way, you will never lose track of your finished versions either.
Usually, you won't be able to tell what commit or tag a generated PDF belongs to once it is in circulation.
But hard-printing that, we can always go back to the very specific commit it belongs to.

\section{Random Thoughts}

\begin{itemize}
	\item If you don't intend to allow \verb|\par| in your custom macros, use the starred variant of \verb|\newcommand|.
	\item Put \verb|{braces}| around anything where it is applicable.
	That includes sub- and superscripts that consist of only one character.
	\verb|x^2| will work, but \verb|x^{2}| will ensure nothing breaks down the line, \iecfeg{e.g.}\ when automatically replacing stuff.
	Superfluous braces don't hurt, but their lack can bite you.
\end{itemize}
