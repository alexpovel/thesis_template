\chapter{Style}

These are just random thoughts on style and design of documents, primarily technical publications of all kinds.
They are a collection of experience, but also aspects taken from various literature on the topic.
The available literature is positively exhausting, so there is no point digging into that here.
Take to it if you are truly interested; the content presented here is on a beginner level.

Of course, anything is subjective here.
When one speaks of \enquote{rules} in style and design, of course what they are really saying is \enquote{\SI{99}{\percent} of people agree on this}.
Many people make an effort to be the other \SI{1}{\percent} (guilty!).
Chances are, it will not come out looking right and you'll be back to the drawing board anyway.

\section{Conventions}
Less is more: hack and alter as little as possible.
Armies of professionals have arrived at today's conventions over centuries.
One example that breaks with what is usually found in documents are the page headers of this document.
The header height is increased beyond its standard (using the fabulous \package{koma-script}) and a vertical rule was inserted.
I like it: a touch different and modern, yet not in the way of anything.
If you don't, feel free to tear it down in the preamble.

Less is more: use the least ink possible to get your point across.
Any more is only noise.
For this, compare \cref{tab:main_font_examples} with its not-so-blessed twin, \cref{tab:main_font_examples_ugly}.
Another observation there: \textbf{if in doubt, left-align}.
If you don't have an \emph{actual reason} to center or right-align, just don't.

\begin{table}
	\ttabbox{%
		\caption%
		{%
			Awful version of \cref{tab:main_font_examples}%
		}%
		\label{tab:main_font_examples_ugly}%
	}%
	{%
		\begin{tabular}{|c||c|}% Use @{} to remove white space from sides (since braces are empty)
		\hline
			\textit{Feature} & \textit{Sample Text}\\
		\hline
		\hline
			Regular & \sampletext\\
		\hline
			\textbf{Bold} & \textbf{\sampletext}\\
		\hline
			\textit{Italics} & \textit{\sampletext}\\
		\hline
			\textbf{\textit{Bold Italics}} & \textbf{\textit{\sampletext}}\\
		\hline
			\textsc{Small Capitals} & \textsc{\sampletext}\\
		\hline
			\textbf{\textsc{Bold SC}} & \textbf{\textsc{\sampletext}}\\
		\hline
			\textit{\textsc{Italics SC}} & \textit{\textsc{\sampletext}}\\
		\hline
		\end{tabular}
	}%
\end{table}

An example for a larger table is shown in \cref{tab:fuel_composition}.
One key aspect there: column-types provided by \package{siunitx}.
These automatically apply \verb|\num| to each entry, which in turn allows easy printing of things like \enquote{\num{-3.23e-5}}.
Use \verb|*x{y}| to print column-type \verb|y| (\iecfeg{i.e.}\ \verb|c|) \verb|x| times.
No need for tedious repetition.
Decimal places are accounted for and aligned by.

\begin{landscape}
\begin{table}
	\crefname{equation}{eq.}{eqs.}% Keep this change scoped to the environment
	\Crefname{equation}{Eq.}{Eqs.}%
	\footnotesize%
	\ttabbox%
	{%
		\caption[Fuel compositions]
		{%
			Compositions and properties of fuels, as used in \Cref{eq:tikz_in_text}%
		}%
		\label{tab:fuel_composition}%
	}%
	{%
		\begin{tabular}{@{}p{5em}*7{S[table-format=1.4]}S[table-format=4.0]S[table-format=2.1]@{}}
			\toprule
			Name & \multicolumn{7}{c}{{Mass fraction \(\gls{massfr}\)}} & {\(\gls{density}\)} & {\(\gls{heatingvalue}_{\gls{inferior}}\)} \\
			& \multicolumn{7}{c}{{[--]}} & \(\qty[\si{\kilogram\per\meter\cubed}]\) & \(\qty[\si{\mega\joule\per\kilogram}]\) \\
			\cmidrule(lr){2-8}
			& \chcpd{C} & \chcpd{H} & \chcpd{S} & \chcpd{O} & \chcpd{N} & \chcpd{H2O} & {Ash} & & \\
			\midrule
			Diesel\mpfootnotemark[1] & 0.8600 & 0.1320 & 0.0060 & \multicolumn{2}{c}{\num{0.0020}\mpfootnotemark[2]} & {n.a.} & {n.a.} & 840 & 42.7 \\
			Oil EL\mpfootnotemark[1] & 0.8570 & 0.1310 & 0.0100 & \multicolumn{2}{c}{\num{0.0020}\mpfootnotemark[2]} & {n.a.} & {n.a.} & 840 & 42.7 \\
			Oil H\mpfootnotemark[1] & 0.8490 & 0.1060 & 0.0350 & \multicolumn{2}{c}{\num{0.0100}\mpfootnotemark[2]} & {n.a.} & {n.a.} & 980 & 40.0 \\
			\Gls{marine_diesel_oil}\mpfootnotemark[3] & {n.a.} & {n.a.} & 0.0150 & {n.a.} & {n.a.} & 0.0030\mpfootnotemark[4] & 0.0001 & 900 & {n.a.} \\
			\Gls{heavy_fuel_oil}\mpfootnotemark[5] & {n.a.} & {n.a.} & 0.0350\mpfootnotemark[6]& {n.a.} & {n.a.} & 0.0050\mpfootnotemark[4] & 0.0015 & 1010 & {n.a.} \\
			\addlinespace
			Light\mpfootnotemark[7] & 0.8600 & 0.1320 & 0.0060 & 0.0010 & 0.0010 & 0 & 0 & 840 & {\Cref{eq:boie}} \\
			Medium\mpfootnotemark[7]& 0.8530 & 0.1269 & 0.0150 & 0.0010 & 0.0010 & 0.0030 & 0.0001 & 900 & {\Cref{eq:boie}} \\
			Heavy\mpfootnotemark[7] & 0.8460 & 0.1025 & 0.0350 & 0.0050 & 0.0050 & 0.0050 & 0.0015 & 1010 & {\Cref{eq:boie}} \\
			\bottomrule
		\end{tabular}%
		\footnotetext[1]{{\autocite[634]{baehr_thermodynamik:_2016}; see also \autocite[L70]{dubbel_taschenbuch_2007} and \autocite[97]{mollenhauer_handbuch_2007}.}}%
		\footnotetext[2]{{Given as a sum \(\gls{massfr}_{\chcpd{O}}+\gls{massfr}_{\chcpd{N}}\).}}%
		\footnotetext[3]{{\Gls{marine_diesel_oil}, max. values of specification \emph{DMB}, \autocite{international_organization_for_standardization_iso_2017}.}}%
		\footnotetext[4]{{Given as a volume fraction, assumed equal to mass fraction.}}%
		\footnotetext[5]{{\Gls{heavy_fuel_oil}, max. values of specification \emph{RMK}, \autocite{international_organization_for_standardization_iso_2017}.}}%
		\footnotetext[6]{{\Gls{int_mar_org} level prior to \num{2020}.}}%
		\footnotetext[7]{{Derived, \emph{virtual} fuels for simulation.}}%
	}%
\end{table}%
\end{landscape}

\section{Sectioning}\label{ch:sectioning}
% There has to be content here
\subsection{Explanation}\label{ch:sectioning_explanation}
Between any two sectional commands (from \verb|\part| down), there always has to be \emph{some} content.
Anything else looks off, \iecfeg{cf.}\ \cref{ch:sectioning,ch:sectioning_explanation}.
At least, tell the reader what the following hierarchical level contains for them.

Further, three numbered levels of hierarchy are enough.
In \package{koma-script} reports (\verb|scrreprt|), these are chapters, sections and subsections.

\subsubsection{Deeper!}
Deeper sectioning is fine, but it should not be numbered or occur in the table of contents.
Notice how, without specifying anything, \package{koma-script} abides by that automatically.
This is despite using \verb|\subsubsection{Deeper!}| as opposed to \verb|\subsubsection*|, its explicitly unnumbered counterpart.

Also notice how between \verb|\section{Sectioning}|, which produced \cref{ch:sectioning}, and \verb|\subsubsection{Deeper!}|, there was no sub-section.
Usually, you would want a clean cascade with no jump in the hierarchy levels.
If you catch yourself jumping, it is also a sign there is something off with your actual content structure and thought process.
That being said, I sometimes still do it because of reasons.
Your call, just be aware.

Amazingly, \LaTeX{} has many ways where it does not let you do something easily, or where commands look off.
\textbf{Never fight \LaTeX{}} in that; you will lose, it will win.
When something is awkward to do or requires a lot of manual labor, you are probably doing it wrong and there will be an easier way.
Most often, that comes in the form of packages.
Being a fully-fleshed programming language under the hood, \emph{any} repetition in \LaTeX{} should make you stop for a second and wonder about another way.
That may lead to you discovering sub-optimal approaches and code.
Just as often, it won't lead anywhere and trying to force minimal code and maximum optimization will not be economical.

\paragraph{Paragraphs}
They are a nice touch, introducing a new, distinct paragraph/idea/thought without being too loud about it.

That being said, leave a blank line in the source code for regular paragraphs.
Use either vertical spacing between paragraphs or indentation (like here).
Do not use both.

\noindent Each new train of thought should go into its own paragraph.
This paragraph is not indented, as was intended by manually calling \verb|\noindent|.
In regular text and documents, there are very few reasons that should ever occur.
Again, let \LaTeX{} do its thing.
Again, if you find yourself typing \verb|\noindent| all the time, there will be something wrong.

\section{Other stuff}\label{ch:other_stuff}
This \textcolor{mRed}{\lcnamecref{ch:other_stuff}} is about auxiliary stuff.
Notice \color{mRed} \verb|\lcnamecref| \color{black} in the previous sentence: it references (in \textit{l}ower-\textit{c}ase) the name of the thing you are passing it.
If this \lcnamecref{ch:other_stuff} ever changes to something else (chapter, \dots), it is updated automatically.

Now for the stuff:
\begin{itemize}
	\item Small spaces are produced by the usual back-slash, followed by an actual space: \verb|\ |.
	Much like physical quantities produced by \package{siunitx} (\SI{30.2e2}{\newton}) and its small spaces, they are nice to have.
	When \LaTeX{} sees any dot \verb|.| with a space behind it, it sees that as the end of the sentence.
	When it isn't, like for Prof.\ Very Important, we use an explicit small space instead.
	
	A second important application would be abbreviations, \iecfeg{i.e.}\ \textleftarrow{} these little things.
	
	\textbf{In the name of all that is holy, never type units, numbers or quantities out by yourself, manually.}
	Use \package{siunitx} for that.
	It will insert a small space between the number and the unit.
	No spaces or full spaces are wrong and painful to read.
	Lastly, even numbers should always go into \verb|\num|.
	Compare 47 sheep versus \num{47} sheep.
	Different fonts are used: the regular text font, for which we specified \verb|OldStyle|, \iecfeg{i.e.}\ hanging numbers, the the math font.
	Other features are global control over decimal and group/thousands separators, how to handle scientific notation (\num{1e1}) and \emph{much} more.
	As always, the manual will be your reference and friend.
	\item Use \verb|---| in the source code to have an em-dash---perhaps even without surrounding spaces?
	That is probably up to your choice.
	What is not, because it is largely agreed on, is to use \verb|--| -- an en-dash -- instead.
	It is reserved for number ranges.
	Lastly, a single hy-phen is reserved for just that --- hyphenation.
	\item A generally important concept in \LaTeX{}:
	\textbf{write what you \emph{mean}, not what produces the desired output}.
	Do \emph{emphasized} text and \textit{emphasized}, but in italics, text look the same?
	They certainly do, usually.
	
	Yet, what did we \emph{mean} to be doing?
	We \emph{mean} to \emph{emphasize}; italic text is just what it happens to look like now, but it is not the \emph{meaning}.
	For example, we could later decide to redefine emphasized text to bold.
	If you previously did not differentiate strictly enough between \verb|\emph| and \verb|\textit|, you are in for a bad time.
	\item Indices are upright!
	They are text.
	If text occurs in math mode, that is actual text, too.
	Consider
	\begin{equation*}
		MEAN_{sample} \neq 2\eqcomma{}
	\end{equation*}
	which looks stupid.
	We can \verb|\DeclareMathOperator{\examplemean}{MEAN}|, then use \verb|\text| for the subscript:
	\begin{equation}\label{eq:eqend}
		\examplemean_{\text{sample}} \neq 2 \eqend{}
	\end{equation}
	\item In \cref{eq:eqend}, and anywhere else, the equation is part of the sentence.
	It is just another language, which can and will be read as a natural part of the surrounding text.
	As such, it should contain punctuation marks.
	So that now, having considered that we have the amazing result of
	\begin{equation}
		1 \neq 2 \eqcomma{}
	\end{equation}
	we have achieved better overall reading flow.
\end{itemize}