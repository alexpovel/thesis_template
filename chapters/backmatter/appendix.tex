% !TeX root = Master_Thesis.tex
% !TeX spellcheck = en_GB
% !TeX program = lualatex

\begin{appendices}

\chapter{Appendices}\label{ch:appendices}

\section{Parameters}\label{ch:appendices_modelling}

\begin{table}[ht]
	\ttabbox
	{%
		\caption[Sutherland equation parameters]%
		{\name{Sutherland}{William} equation parameters, \iecfeg{cf.}\ \cref{eq:sutherland_simple}}
		\label{tab:sutherland}%
	}%
	{%
	\begin{tabular}{@{}lcc*2{S[table-format=2.1]}@{}}
		\toprule
		Component       &      Symbol       & {\(T\mr{S}\) \(\qty[\si{\kelvin}]\)} & {\(\gls{dynvisc}{}_0\)\mpfootnotemark[4] \(\qty[\si{\micro\pascal\second}]\)} & {\(T_0\)\mpfootnotemark[4] \(\qty[\si{\kelvin}]\)} \\ \midrule
		Oxygen          &    \chcpd{O2}     &       127.0\mpfootnotemark[1]        & 20.8                                                                          & 300                                                \\
		Nitrogen        &    \chcpd{N2}     &        111\mpfootnotemark[1]         & 17.9                                                                          & 300                                                \\
		Argon           &    \chcpd{Ar}     &        170\mpfootnotemark[2]         & 22.9                                                                          & 300                                                \\
		Carbon Dioxide  &    \chcpd{CO2}    &        240\mpfootnotemark[1]         & 15.0                                                                          & 300                                                \\
		Water Vapour    & \chcpd{H2O_{(g)}} &        549\mpfootnotemark[3]         & 10.0                                                                          & 300                                                \\
		Sulphur Dioxide &    \chcpd{SO2}    &        416\mpfootnotemark[1]         & 12.9                                                                          & 300                                                \\ \bottomrule
	\end{tabular}%
	\footnotetext[1]{\textcite[A5]{crane_co._flow_1982}}%
	\footnotetext[2]{\textcite[225]{chapman_mathematical_1953}}%
	\footnotetext[3]{\cref{eq:vogel} with \(T\mr{boiling} = T\mr{boiling}\qty(\SI{1}{\atmosphere}) = \SI{373.15}{\kelvin}\)}%
	\footnotetext[4]{\textcite{lide_crc_2005}, Table \enquote{Viscosity of Gases}}%
	}%
\end{table}%

\begin{table}[ht]
\ttabbox
{%
	\caption{Sets of ambient states}%
	\label{tab:ambient_states}%
}%
{%
\begin{tabular}{@{}lS[table-format=3.2]S[table-format=1.5e5]l@{}}
\toprule
	Organization & {\(T\mr{\gls{ambient}}\) [\si{\kelvin}]} & {\(p\mr{\gls{ambient}}\) [\si{\pascal}]} & Reference\\
\midrule
	\glsxtrshort{nist} & 293.15 & 1.01325e5 & \autocite{phillips_2016_2016,nist_constants_2016}\\
	\glsxtrshort{iupac} & 273.15 & 1.00000e5 & \autocite{iupac_compendium_1997}\\
	\glsxtrshort{spe} & 288.15 & 1.00000e5 & \autocite{society_of_petroleum_engineers_si_1984}\\
\bottomrule
\end{tabular}
}%
\end{table}

\begin{table}[ht]
	\renewcommand{\arraystretch}{1.5}%more vertical space, creating a cleaner, more relaxed and prettier appearance
	\small%
	\ttabbox
	{%
		\caption[Calibration Parameters]{List of calibration parameters used in simulation}%
		\label{tab:calibration_parameters}%
	}%
	{%
		\begin{tabular}{@{}>{$}c<{$}*2{l}*2{c}@{}}
			\toprule%
			\multicolumn{3}{c}{\text{Parameter}} & \multicolumn{2}{c}{Introduction}\\
			\cmidrule(r){1-3}\cmidrule(l){4-5}
			\text{Symbol} & calibrates\dots & affects\dots & Eq. & Page\\
			\midrule%
			k\mr{\gls{throat}} & Throat Area & Choke Line & \labelcref{eq:blockage_throat} & \labelcpageref{eq:blockage_throat}\\
			B^{\ast} & Blockage & Throughput & \labelcref{eq:blockage_with_calibration} & \labelcpageref{eq:blockage_with_calibration}\\
			D & Diffusion Limit & Surge Line & \labelcref{eq:diffusion_factor_limit} & \labelcpageref{eq:diffusion_factor_limit}\\
			k\mr{\gls{diffuser}} & Diffuser Friction & Diffuser Performance & \labelcref{eq:stanitz_equi_mer} & \labelcpageref{eq:stanitz_equi_mer}\\
			P\mr{\gls{wastegate}},\; I\mr{\gls{wastegate}} & Wastegate & Boost Pressure & \labelcref{eq:wg_pid_parameters} & \labelcpageref{eq:wg_pid_parameters} \\
			\bottomrule
		\end{tabular}
	}%
\end{table}%

\begin{landscape}
\begin{table}
\crefname{equation}{eq.}{eqs.}%
\Crefname{equation}{Eq.}{Eqs.}%
\footnotesize%
\ttabbox%
{%
\caption[Fuel compositions]
{%
	Compositions and properties of fuels, as used in \Cref{eq:definition_fuel_composition}
}%
\label{tab:fuel_composition}%
}%
{%
\begin{tabular}{@{}p{5em}*7{S[table-format=1.4]}S[table-format=4.0]S[table-format=2.1]@{}}
\toprule
	Name & \multicolumn{7}{c}{{Mass fraction \(\gls{massfr}\)}} & {\(\gls{density}\)} & {\(H\mr{\gls{inferior}}\)} \\
	& \multicolumn{7}{c}{{[--]}} & \(\qty[\si{\kilogram\per\meter\cubed}]\) & \(\qty[\si{\mega\joule\per\kilogram}]\) \\
\cmidrule(lr){2-8}
	& \chcpd{C} & \chcpd{H} & \chcpd{S} & \chcpd{O} & \chcpd{N} & \chcpd{H2O} & {Ash} & & \\
\midrule
	Diesel\mpfootnotemark[1] & 0.8600 & 0.1320 & 0.0060 & \multicolumn{2}{c}{\num{0.0020}\mpfootnotemark[2]} & {n.a.} & {n.a.} & 840 & 42.7 \\
	Oil EL\mpfootnotemark[1] & 0.8570 & 0.1310 & 0.0100 & \multicolumn{2}{c}{\num{0.0020}\mpfootnotemark[2]} & {n.a.} & {n.a.} & 840 & 42.7 \\
	Oil H\mpfootnotemark[1] & 0.8490 & 0.1060 & 0.0350 & \multicolumn{2}{c}{\num{0.0100}\mpfootnotemark[2]} & {n.a.} & {n.a.} & 980 & 40.0 \\
	\gls{mdo}\mpfootnotemark[3] & {n.a.} & {n.a.} & 0.0150 & {n.a.} & {n.a.} & 0.0030\mpfootnotemark[4] & 0.0001 & 900 & {n.a.} \\
	\gls{hfo}\mpfootnotemark[5] & {n.a.} & {n.a.} & 0.0350\mpfootnotemark[6]& {n.a.} & {n.a.} & 0.0050\mpfootnotemark[4] & 0.0015 & 1010 & {n.a.} \\
\addlinespace
	Light\mpfootnotemark[7] & 0.8600 & 0.1320 & 0.0060 & 0.0010 & 0.0010 & 0 & 0 & 840 & {\Cref{eq:boie}} \\
	Medium\mpfootnotemark[7]& 0.8530 & 0.1269 & 0.0150 & 0.0010 & 0.0010 & 0.0030 & 0.0001 & 900 & {\Cref{eq:boie}} \\
	Heavy\mpfootnotemark[7] & 0.8460 & 0.1025 & 0.0350 & 0.0050 & 0.0050 & 0.0050 & 0.0015 & 1010 & {\Cref{eq:boie}} \\
\bottomrule
\end{tabular}%
\footnotetext[1]{{\autocite[634]{baehr_thermodynamik:_2016}; see also \autocite[L70]{dubbel_taschenbuch_2007} and \autocite[97]{mollenhauer_handbuch_2007}.}}%
\footnotetext[2]{{Given as a sum \(\gls{massfr}_{\chcpd{O}}+\gls{massfr}_{\chcpd{N}}\).}}%
\footnotetext[3]{{\Gls{mdo}, max. values of specification \emph{DMB}, \autocite{international_organization_for_standardization_iso_2017}.}}%
\footnotetext[4]{{Given as a volume fraction, assumed equal to mass fraction.}}%
\footnotetext[5]{{\Gls{hfo}, max. values of specification \emph{RMK}, \autocite{international_organization_for_standardization_iso_2017}.}}%
\footnotetext[6]{{\gls{imo} level prior to \num{2020}.}}%
\footnotetext[7]{{Derived, \emph{virtual} fuels for simulation.}}%
}%
\end{table}%

\begin{table}
\newcommand*{\temptablewidth}{8em}
\footnotesize
\ttabbox
{%
\caption[Shomate equation coefficients]
{%
	Coefficients \(A\dots E\) for the \name{Shomate}{C. Howard} \cref{eq:definition_shomate}, \autocite{national_institute_of_standards_and_technology_argon_2017,national_institute_of_standards_and_technology_carbon_2017,national_institute_of_standards_and_technology_nitrogen_2017,national_institute_of_standards_and_technology_oxygen_2017,national_institute_of_standards_and_technology_sulphur_2017,national_institute_of_standards_and_technology_water_2017}%
}%
\label{tab:shomate_parameters}
}
{%
\begin{tabular}{@{}lrrS[table-format=2.5]*4{S[table-format=-2.6e-1]}@{}}
\toprule
	Gas & \multicolumn{2}{c}{T [\si{\kelvin}]} &  \multicolumn{5}{c}{Parameter}\\
	\cmidrule(r){2-3}\cmidrule(l){4-8}
	& \multicolumn{1}{l}{from} & \multicolumn{1}{l}{to} & {A} & {B} & {C} & {D} & {E} \\
\midrule
	\multirow{2}{\temptablewidth}{Oxygen (\chcpd{O2})} & \num{100} & \num{700} & 31.32234 & -20.23531 & 57.86644 & -36.50624 & -0.007374 \\
	& \num{700} &\num{2000} & 30.03235 & 8.772972 & -3.988133 & 0.788313 & -0.741599 \\
\addlinespace
	\multirow{2}{\temptablewidth}{Nitrogen (\chcpd{N2})} & \num{100} & \num{500} & 28.98641 & 1.853978 & -9.647459 & 16.63537 & 0.000117 \\
	& \num{500} & \num{2000} & 19.50583 & 19.88705 & -8.598535 & 1.369784 & 0.527601 \\
\addlinespace
	Argon (\chcpd{Ar}) & \num{298} &\num{6000} & 20.78600 & 2.825911e-7 & -1.464191e-7 & 1.092131e-8 & -3.661371e-8 \\
\addlinespace
	\multirow{2}{\temptablewidth}{Carbon Dioxide (\chcpd{CO2})} & \num{298} & \num{1200} & 24.99735 & 55.18696 & -33.69137 & 7.948387 & -0.136638 \\
	& \num{1200} & \num{6000} & 58.16639 & 2.720074 & -0.492289 & 0.038844 & -6.447293 \\
\addlinespace
	\multirow{2}{\temptablewidth}{Water (\chcpd{H2O_{(g)}})} & \num{500} & \num{1700} & 30.09200 & 6.832514 & 6.793435 & -2.534480 & 0.082139 \\
	& \num{1700} & \num{6000} & 41.96426 & 8.622053 & -1.499780 & 0.098119 & -11.15764 \\
\addlinespace
	\multirow{2}{\temptablewidth}{Sulphur Dioxide (\chcpd{SO2})} & \num{298} & \num{1200} & 21.43049 & 74.35094 & -57.75217 & 16.35534 & 0.086731 \\
	& \num{1200} & \num{6000} & 57.48188 & 1.009328 & -0.076290 & 0.005174 & -4.045401 \\
\bottomrule
\end{tabular}%
}%
\end{table}

\begin{table}%
\small
\ttabbox%
{%
	\caption[Engines used in simulation]{Parameters of engines as illustrated in \cref{ch:engine_parameters}. Two-stroke working cycles equal \(a=1\), accordingly \(a=2\) for four-stroke. Intake and exhaust manifold sizes are assumed equal}%
	\label{tab:engine_parameters}%
}%
{%
\begin{tabular}{@{}
	M{3em}
	c
	S[table-format=3.0]
	S[table-format=2.1]
	S[table-format=1.0]
	S[table-format=4.1]
	S[table-format=1.1]
	S[table-format=4.0]
	S[table-format=1.0]
	S[table-format=3.0]
	S[table-format=1.3]@{}}
\toprule
	Model & Ref. & {\(P\mr{\gls{cylinder}}\)} & {\(V\mr{\gls{cyldis}}\)} & {\(z\mr{\gls{cylinder}}\)} & {\(I\mr{\gls{engine}}\)} & {\(p\mr{\gls{air},\gls{ratedeng}}\)} & {\(n\mr{\gls{ratedeng}}\)} & {\(a\)} & {\(b\mr{\gls{engine}}\)\mpfootnotemark[1]} & {\(V\mr{\gls{inlet}}\)/\(V\mr{\gls{exhaust}}\)\mpfootnotemark[2]}\\
%\cmidrule(r){1-2}\cmidrule(l){3-11}
	& & {[\si{\kilo\watt}]} & {[\si{\liter}]} & {[--]} & {[\si{\kilogram\meter\squared}]} & {[\si{\bar}]} & {[\si{\per\minute}]} & {[--]} & {[\si{\gram\per\kWh}]} & {[\si{\meter\cubed}]}\\
\midrule
	\textsc{\textbf{MaK} 9M43C} & \autocite{caterpillar_motoren_gmbh_m_2008} & 900 & 88.6 & 9 & 4760 & 3.1 & 500 & 2 & 177 & 6.48\\
	\textsc{\textbf{MaK} 6M20C} & \autocite{caterpillar_motoren_gmbh_m_2012} & 170 & 9.4 & 6 & 86.4 & 3.3 & 900 & 2 & 187 & 0.816\\
	\textsc{\textbf{MAN} L16/24} & \autocite{man_diesel_and_turbo_l16/24_2018} & 95 & 4.83 & 6 & 54 & 4.1 & 1000 & 2 & 195 & 0.456\\
\bottomrule
\end{tabular}
\footnotetext[1]{Based on \(H\mr{\gls{inferior}} = \SI{42.7}{\mega\joule\per\kilogram}\).}%
\footnotetext[2]{From \(V\approx P\mr{\gls{ratedeng}} \cdot \SI{0.8}{\meter\cubed\per\mega\watt}\), \iecfeg{cf.}\ \autocite[13]{sell_unterschiede_2003}.}%
}%
\end{table}

\begin{table}\ContinuedFloat*
\small
\ttabbox%
{%
\caption[Compressor geometry parameters]
{%
	Compressor geometry parameters.
	Lengths rounded to four significant digits%
}%
\label{tab:compressor_geometry}
}%
{%
\censorbox{%
\begin{tabular}{@{}l*5{S[table-format=1.4]}*2{S[table-format=3.1]}*2{S[table-format=2.0]}*3{S[table-format=1.4]}@{}}
\toprule
	& \multicolumn{5}{c}{{Radii \& Heights\mpfootnotemark[1] [\si{\meter}]}} & \multicolumn{2}{c}{{Metal Angles\mpfootnotemark[2] [\si{\degree}]}} & \multicolumn{2}{c}{{Blades}} & \multicolumn{2}{c}{Empirical Blockage\mpfootnotemark[3]} & {Gap\mpfootnotemark[4]}\\
\cmidrule(lr){2-6}\cmidrule(lr){7-8}\cmidrule(lr){9-10}\cmidrule(lr){11-12}\cmidrule(l){13-13}
	& {Hub} & {Inner} & {Outer} & \multicolumn{2}{c}{{Diffuser}} & {Inducer} & {Discharge} & {Split} & {All} & {Inlet} & {Outlet} & {--}\\
\cmidrule(lr){5-6}
	& {\(r\mr{\gls{hub}}\)} & {\(r\mr{\gls{inside}}\)} & {\(r\mr{\gls{outside}}\)} & {\(r\mr{\gls{diffuser}}\)} & {\(h\mr{\gls{diffuser}}\)} & {\(\gls{angrel}_{1}\)} & {\(\gls{angrel}_{2}\)} & {\(Z_1\)} & {\(Z_2\)} & {\(b_1\)} & {\(b_2\)} & {\(s\)} \\
\midrule
	VATN & 0.0111 & 0.0212 & 0.0310 & 0.0500 & 0.0063 & 143.5 & 120 & {--} & 12 & {--} & {--} & 0.010\\
	T03 & 0.0078 & 0.0203 & 0.0300 & 0.0500 & 0.0046 & 143.5 & 120 & {--} & 18 & {--} & {--} & 0.010\\
	K26 & 0.0068 & 0.0210 & 0.0330 & 0.0700 & 0.0072 & 142.5 & 120 & {--} & 12 &{--} & {--} & 0.010\\
\addlinespace
	ZR1 & 0.0130 & 0.0433 & 0.0675 & 0.1030 & 0.0074 & 136.2 & 150 & 6 & 12 & 0.0012 & 0.0014 & 0.010\\
	ZR3 & 0.0210 & 0.0650 & 0.1025 & 0.1625 & 0.0122 & 135.4 & 150 & 8 & 16 & 0.0013 & 0.0024 & 0.010\\
	ABB & 0.0236 & 0.0679 & 0.1000 & 0.1324 & 0.0116 & 129.0 & 150 & 8 & 16 & 0.0013 & 0.0024 & 0.010\\
\bottomrule
\end{tabular}
\footnotetext[1]{See \cref{fig:euler_momentum_dimensions}.}
\footnotetext[2]{Following the conventions from \cref{ch:fundamentals,fig:angle_definitions}.}
\footnotetext[3]{As determined by \textcite{lyssewski_weiterentwicklung_2019}.}
\footnotetext[4]{Refer to \autocites[114]{joos_turbinen_2016}[253\psq]{storer_tip_1991}[7]{gbadebo_interaction_2006}[13]{povel_one-dimensional_2018}.}
}%
}%
\end{table}

\begin{table}\ContinuedFloat
\small
\ttabbox%
{%
\caption[Turbine geometry parameters]
{%
	Turbine geometry parameters%
}%
\label{tab:turbine_geometry}
}%
{%
\censorbox{%
\begin{tabular}{@{}l*4{S[table-format=1.4]}*1{S[table-format=2.0]}*1{S[table-format=3.1]}*1{S[table-format=2.0]}*2{S[table-format=1.4]}S[table-format=1.4]c@{}}
\toprule
	& \multicolumn{4}{c}{{Radii \& Heights\mpfootnotemark[1] [\si{\meter}]}} & \multicolumn{2}{c}{{Metal Angles\mpfootnotemark[2] [\si{\degree}]}} & {Blades} & \multicolumn{2}{c}{Empirical Blockage\mpfootnotemark[3]} & \multicolumn{2}{c}{{Other}}\\
\cmidrule(lr){2-5}\cmidrule(lr){6-7}\cmidrule(lr){8-8}\cmidrule(lr){9-10}\cmidrule(l){11-12}
	& {Hub} & {Inner} & {Outer} & {Inlet} & {Inlet} & {Discharge} & {--} & {Inlet} & {Outlet} & {\glsxtrshort{ar}\mpfootnotemark[4] [\si{\meter}]} & {\glsxtrshort{vs}\mpfootnotemark[5]}\\
	& {\(r\mr{\gls{hub}}\)} & {\(r\mr{\gls{inside}}\)} & {\(r\mr{\gls{outside}}\)} & {\(h\)} & {\(\gls{angrel}_{1}\)} & {\(\gls{angrel}_{2}\)} & {\(Z\)} & {\(b_1\)} & {\(b_2\)} & {--} & {--}\\
\midrule
	VATN & 0.0078 & 0.0239 & 0.0300 & 0.0075 & 90 & 147.0 &  12 & {--} & {--} & {--} & {--}\\
	T03 & 0.0084 & 0.0245 & 0.0315 & 0.0105 & 90 & 156.0 &  11 & {--} & {--} & {--} & {--}\\
	K26 & 0.0108 & 0.0273 & 0.0320 & 0.0102 & 90 & 140.0 &  12 &{--} & {--} & {--} & {--}\\
\addlinespace
	ZR1 & 0.0250 & 0.0565 & 0.0615 & 0.0165 & 90 & 128.0 & 11 & 0.0012 & 0.0031 & 0.0285 & N\\
	ZR3 & 0.0290 & 0.0765 & 0.0900 & 0.0230 & 90 & 131.7 & 13 & 0.0012 & 0.0034 & 0.0420 & N\\
	ABB & 0.0314 & 0.0888 & 0.0926 & 0.0281 & 90 & 131.7 & 12 & 0.0012 & 0.0034 & {--} & Y\\
\bottomrule
\end{tabular}
\footnotetext[1]{See \cref{fig:euler_momentum_dimensions}.}
\footnotetext[2]{Following the conventions from \cref{ch:fundamentals,fig:angle_definitions}.}
\footnotetext[3]{As determined by \textcite{lyssewski_weiterentwicklung_2019}.}
\footnotetext[4]{\censor{\Gls{ar} of volute area against its radius (considered constant), \iecfeg{cf.}\ \autocite{lyssewski_weiterentwicklung_2019}.}}
\footnotetext[5]{\censor{Presence of a \gls{vs} (Yes/No). In case of its presence, the \gls{ar} is irrelevant, \iecfeg{cf.}\ \autocite{lyssewski_weiterentwicklung_2019}.}}
}%
}%
\end{table}
\end{landscape}

\clearpage

\section{Miscellaneous}\label{ch:misc}

\textit{This \lcnamecref{ch:misc} is a collection of all content that fit neither the main document, nor another appendix category.}

\subsection{Efflux Function}\label{ch:efflux_function}

Taking the absolute system as an example, inserting \cref{eq:rho_rearranged,eq:first_law_velocity,eq:isentropic_relation_temperature} as well as the relation
\begin{equation}
	\vectcomp{\gls{vela}}{}{\gls{meridional}} = \sin\gls{angabs}{}\vectcomp{\gls{vela}}{}{}
\end{equation}
into \cref{eq:mass_conservation} yields
\begin{equation}
\begin{aligned}
\dot{m} &= \frac{p_0^{\frac{\gls{adexp}-1}{\gls{adexp}}}}{T_0 R}p_1^{1/\gls{adexp}}  \sin\alpha A \sqrt{2c_p\qty(T_0 - T_1)} \\
&= \frac{p_0^{\frac{\gls{adexp}-1}{\gls{adexp}}}}{T_0 R} p_1^{1/\gls{adexp}} \sin\alpha A\sqrt{2c_pT_0\qty(1 - \qty(\frac{p_1}{p_0})^{(\gls{adexp}-1)/\gls{adexp}})} \\
&= \underbrace{\frac{p_0^{\frac{\gls{adexp}-1}{\gls{adexp}}}}{T_0 R} \sin\alpha A \sqrt{2c_pT_0}}_{\mathrm{\const}} p_1^{1/\gls{adexp}} \sqrt{1 - \qty(\frac{p_1}{p_0})^{(\gls{adexp}-1)/\gls{adexp}}}
\end{aligned}
\end{equation}

Defining a pressure ratio
\begin{equation}
\Pi_i \coloneq \frac{p_i}{p_0}
\end{equation}
so that \(\Pi_1 = p_1/p_0\) and inserting \cref{eq:ideal_gas_law_mass} as well as the relation \(c_p = R\gls{adexp}/(\gls{adexp} - 1)\) finally yields
\begin{equation}\label{eq:mass_flux_through_nozzle}
\begin{aligned}
\dot{m} &= \frac{p_0}{T_0 R}  \sin\alpha  A  \sqrt{2c_pT_0} \Pi_1^{1/\gls{adexp}}  \sqrt{1 - \Pi_1^{(\gls{adexp}-1)/\gls{adexp}}}\\
&= \underbrace{\sin\alpha  A  \sqrt{2p_0 \gls{density}_0}}_{\mathrm{\const}}  \underbrace{\sqrt{\frac{\gls{adexp}}{\gls{adexp} - 1}}  \Pi_1^{1/\gls{adexp}}  \sqrt{1 - \Pi_1^{(\gls{adexp}-1)/\gls{adexp}}}}_{\Psi}\\
&= \sin\alpha  A  p_0 \sqrt{\frac{2}{RT_0}} \Psi
\end{aligned}
\end{equation}
as illustrated in \cref{fig:psi_over_pi_and_adexp}.
Here, \(\Psi\) is a function \(\Psi = \Psi\qty(\gls{adexp},\Pi)\) and since both \(\gls{adexp}\) as well as \(\Pi\)  are naturally constrained such that \(\num{1.3} \leq \gls{adexp} \leq \num{1.7}\) and \(0 \leq \Pi \leq 1\), so is \(\Psi\qty(\gls{adexp},\Pi)\).

The former designation for \(\gls{adexp}\) covers the range of monatomic and polyatomic gases, where for monatomic gases, such a helium, the specific heats ratio is \num{10/6}, and \num{4/3} for polyatomic gases, such as \chcpd{CO2} \autocite{nave_specific_2017}.
Note that in reality, these are temperature-sensitive quantities, \iecfeg{cf.}\ \cref{ch:caloric_properties}.
Due to its non-linearity (and therefore comparatively expensive computation) yet constant nature, \(\Psi\qty(\gls{adexp},\Pi)\) is computed just \emph{once} and the results deposited in a \SimulinkLUT{} object.

Further dissection, \iecfeg{cf.}\ \cref{eq:pi_crit}, yields
\begin{equation}\label{eq:definition_psi_extrema}
\Psi\mr{\gls{maximum}}\qty(\gls{adexp}) = \sqrt{\frac{\gls{adexp}}{\gls{adexp} + 1}}  \qty(\frac{2}{\gls{adexp} + 1})^{\frac{1}{\gls{adexp} - 1}}\qq{\text{at}}\Pi\rvert_{\Psi\mr{\gls{maximum}}} = \Pi\mr{\gls{critical}}\qty(\gls{adexp}) = \qty(\frac{2}{\gls{adexp} + 1})^{\frac{\gls{adexp}}{\gls{adexp} - 1}}
\end{equation}
in
\begin{equation}\label{eq:definition_psi}
\Psi\qty(\gls{adexp},\Pi) = \sqrt{\frac{\gls{adexp}}{\gls{adexp} - 1}}  \Pi_1^{1/\gls{adexp}} \sqrt{1 - \Pi_1^{(\gls{adexp}-1)/\gls{adexp}}}\eqend{}
\end{equation}

In reality however, once the critical pressure ratio is undercut, no further increase in \(\Psi\) and therefore velocity or volumetric flux is observed.
For any pressure ratio \(\Pi < \Pi\mr{\gls{critical}}\), the efflux function\term{function!efflux} remains constant at \(\Psi\mr{\gls{maximum}}\), \iecfeg{cf.}\ \cref{fig:psi_over_pi_and_adexp}.
Truncating the original definition to now reflect the correct, observed physical behaviour yields the piecewise function
\begin{equation}\label{eq:definition_psi_piecewise}
\tcbhighmath{%
	\Psi\qty(\gls{adexp},\Pi) = 
\begin{cases} 
\sqrt{\frac{\gls{adexp}}{\gls{adexp}+1}}  \qty(\frac{2}{\gls{adexp}+1})^{\frac{1}{\gls{adexp}-1}} & \forall \quad 0\leq\Pi<\Pi\mr{\gls{critical}}\\
\sqrt{\frac{\gls{adexp}}{\gls{adexp} - 1}}  \Pi_1^{1/\gls{adexp}}  \sqrt{1 - \Pi_1^{(\gls{adexp}-1)/\gls{adexp}}} & \forall\quad \Pi\mr{\gls{critical}}\leq\Pi\leq 1\eqend{}
\end{cases}%
}%
\end{equation}
\begin{figure}%https://tex.stackexchange.com/questions/457730/3d-piecewise-plot-with-varying-bounds-in-tikz-pgf
	\fcapside[\FBwidth]%
	{%
		\caption[Efflux function over pressure ratio and adiabatic exponent]%
		{%
			Mass flux coefficient \(\Psi\) for an isentropic nozzle.
			Note the infinite gradient as \(\Pi\to 1\)%
		}%
		\label{fig:psi_over_pi_and_adexp}%
		\floatfoot{See \cref{eq:definition_psi_piecewise}}%
	}%
	{%
	\begin{tikzpicture}
	\begin{axis}[
	plotstyleNarrow,
	xlabel={\(\gls{adexp}\) [--]},
	ylabel={\(\Pi\) [--]},
	zlabel={\(\Psi\) [--]},
	view={55}{30},
	]
	\addplot3[thin,opacity=0.6,domain=1.3:1.7,y domain=0.1:1,surf]{psi(x,y)};%this bugs out on occasion. It should work if externalize is ON. If not, add 'samples = 17' or a similiar, lower count than what Tikz would automatically choose. Since the gradient becomes infinity for Pi -> 1, this can also fail. Use a different amount of samples in that case.
	\end{axis}
	\end{tikzpicture}%
	}%
\end{figure}

\subsection{Emissions and Turbocharging}\label{ch:emissions_appendix}

Environmentally, even more factors come into play next to efficiency concerns, as discussed in \cref{ch:multiturbo} on \cpageref{ch:multiturbo}, \autocite[1347]{shahed_parametric_2009}.
\gls{pm} as well as oxides of nitrogen (\chcpd{NO_X}) and sulphur (mainly \chcpd{SO2}) adversely affect the environment, humans and other elements of the biosphere.
Filters are a simple, proven way to reduce \gls{pm}.
Sulphur oxides are decreased if more refined fuel is used, since any elementary sulphur present in the input will inevitably be oxidized and emitted.
If such is the case, scrubbers may be used to clean the exhaust gases after the fact.
Nitrogen oxide generation proves more difficult to avoid.
Granted no presence of elementary nitrogen in the fuel, the combustion air itself delivers the required components.
Three influential parameters,
\begin{enumerate*}
\item higher temperatures,
\item higher oxygen concentrations and
\item longer time frames,
\end{enumerate*}
will shift the balance towards \chcpd{NO_X} formation \autocite[352\psq]{vanderlans_influence_1997}.
In fact, rich combustion inhibits \chcpd{NO_X} generation \autocite[270]{houshfar_experimental_2012}; lean combustion as found in marine \name{Diesel}{Rudolf} engines promotes it.
Secondary measures include \gls{lnt} \autocite{hu_advanced_2006} and \gls{scr} \autocite[121]{stone_introduction_2012}, where the latter is found to be most effective for marine applications \autocite{lamas_emissions_2012}.
As a third option, \gls{egr} may be employed to increase the charge air's specific heat capacity as well as lower its oxygen concentration \autocites[118]{stone_introduction_2012}{shojaeefard_cooled_2013}.
However, this recirculation disturbs the turbocharger flow balance, up to a point where the turbine experiences only half the corrected flow of its compressor \autocite[233]{arnold_single_2008}.
The compressor will have to deliver more fresh-air flow just to upkeep oxygen levels \autocite[2]{kech_turbocharging:_2014}.
At this point, employing multiple turbochargers can alleviate these issues.

\clearpage

\begin{table}
\tiny
\floatbox{table}%
{%https://tex.stackexchange.com/q/295567/120853
\begin{subfloatrow}
\ttabbox[\FBwidth]%
{%
	\begin{tabular}{@{}m{0.25\textwidth}m{0.15\textwidth}@{}}
	\toprule
		Author \& [Work] & Comment\\
	\midrule
		\textcite{bozza_map-based_2011} & ---\\
		\textcite{bozza_numerical_2013} & ---\\
		\textcite{bozza_theoretical_1990} & \textbf{Nozzles}\\
		\textcite{burke_modelling_2014} & ---\\
		\textcite{de_bellis_development_2018} & ---\\
		\textcite{berndt_einfluss_2009} & ---\\
		\textcite{chesse_performance_2000} & ---\\
		\textcite{cornolti_1d_2013} & ---\\
		\textcite{eriksson_modeling_2007} & ---\\
		\textcite{grigoriadis_experimentelle_2008} & ---\\
		\textcite{kech_model-based_2002} & ---\\
		\textcite{lee_simulation-based_2009} & ---\\
		\textcite{leufven_surge_2011} & ---\\
		\textcite{shaaban_part-load_2006} & ---\\
		\textcite{wahlstrom_modelling_2011} & ---\\
		\textcite{watel_matching_2010} & \mtlbsmlnk{}\\
		\textcite{yang_mixed_2010} & ---\\
	\bottomrule
	\end{tabular}
}%
{%
	\caption{Using Maps}%
	\label{tab:works_using_maps}%
}%
\ttabbox[\FBwidth]%
{%
	\begin{tabular}{@{}m{0.25\textwidth}m{0.15\textwidth}@{}}
	\toprule
		Author \& [Work] & Comment\\
	\midrule
		\textcite{beckey_compressor_2011} & using algorithms\\
		\textcite{bergqvist_prediction_2014} & from CFD\\
		\textcite{bohn_modellierung_2002} & ---\\
		\textcite{bolz_critical_2014} & ---\\
		\textcite{elkamel_experimental_2011} & ---\\
		\textcite{ewert_modellierung_2013} & ---\\
		\textcite{freeman_compressor_2011} & \mtlb{} script\\
		\textcite{hansch_untersuchung_2010} & ---\\
		\textcite{harley_inlet_2014} & ---\\
		\textcite{luddecke_engine_2014} & ---\\
		\textcite{kurzke_correlations_2011} & ---\\
		\textcite{schwarz_considerations_2014} & ---\\
	\bottomrule
	\end{tabular}
}%
{%
	\caption{Computing Maps}%
	\label{tab:works_computing_maps}%
}%
\end{subfloatrow}

\begin{subfloatrow}
\ttabbox[\FBwidth]%
{%
	\begin{tabular}{@{}m{0.25\textwidth}m{0.15\textwidth}@{}}
	\toprule
		Author \& [Work] & Comment\\
	\midrule
		\textcite{aungier_mean_1995} & ---\\
		\textcite{casey_method_2013} & Calibrating with Maps\\
		\textcite{chen_one-dimensional_2014} & Turbine\\
		\textcite{erickson_centrifugal_2008} & \fortran{}\\
		\textcite{ewert_modellierung_2013} & ---\\
		\textcite{gong_total_2014} & ---\\
		\textcite{gutierrez_velasquez_one_2010} & \fortran{}, 1D and 3D\\
		\textcite{japikse_turbomachinery_2009} & Overview\\
		\textcite{kamaleshaiah_improved_1988} & ---\\
		\textcite{kessel_modellbildung_2002} & 1- \& 3D\\
		\textcite{lee_dual-stage_2008} & dual-stage; \textbf{\smlnk{}, maps}\\
		\textcite{nakhjiri_vatl_2018} & ---\\
		\textcite{okhuahesogie_1-d_2014} & Differential Evolution\\
		\textcite{pelton_one-dimensional_2007} & ---\\
		\textcite{pixberg_modellbildung_2013} & \textbf{Reservoirs; Nozzles} (p.\ 42)\\
		\textcite{sakellaridis_development_2015} & ---\\
		\textcite{schiffmann_design_2010} & ---\\
		\textcite{schlador_erstellung_2008} & \fortran{}, \textbf{Reservoirs; Models}\\
		\textcite{schur_transient_2013} & \smlnk{}\\
		\textcite{serrano_model_2008} & \textbf{Reservoirs; Nozzles}\\
		\textcite{shaaban_experimental_2004} & Diabatic; Extrapolation\\
		\textcite{stuart_analysis_2014} & cites \autocite{harley_evaluation_2013}\\
		\textcite{taburri_model-based_2012} & \textbf{Reservoirs; Nozzles}\\
		\textcite{uchida_transient_2006} & ---\\
		\textcite{whitfield_preliminary_1990} & Preliminary Design\\
		\textcite{wirz_berechnungsmethode_2017} & \textbf{Reservoirs; Models}\\
		\textcite{xu_empirical_2012} & ---\\
		\textcite{zahn_arbeitsspielaufgeloste_2012} & ---\\
		\textcite{zhuge_development_2009} & ---\\
	\bottomrule
	\end{tabular}
}%
{%
	\caption{Physical (0- \& 1D)}%
	\label{tab:works_physical}%
}%
\ttabbox[\FBwidth]%
{%
	\begin{tabular}{@{}m{0.25\textwidth}m{0.15\textwidth}@{}}
	\toprule
		Author \& [Work] & Comment\\
	\midrule
		\textcite{aungier_mean_1995} & ---\\
		\textcite{doustmohammadi_experimental_2013} & ---\\
		\textcite{el-maksoud_prediction_2012} & ---\\
		\textcites{galvas_analytical_1972}{galvas_fortran_1973} & ---\\
		\textcite{gulich_disk_2003} & Disk Friction\\
		\textcite{gutierrez_velasquez_determination_2017} & ---\\
		\textcite{harley_evaluation_2013} & Recommends \autocite{oh_optimum_1997} \\
		\textcite{mohtar_increasing_2010} & ---\\
		\textcite{nakhjiri_physical_2011} & ---\\
		\textcite{oh_optimum_1997} & ---\\
		\textcite{tacconi_investigation_2018} & Cites \autocite{harley_evaluation_2013}; recommends \autocite{galvas_fortran_1973}\\
		\textcite{schneider_analytical_2015} & 2D\\
		\textcite{solaesa_analytical_2016} & ---\\
	\bottomrule
	\end{tabular}
}%
{%
	\caption{Loss Modelling}%
	\label{tab:works_losses}%
}%
\end{subfloatrow}%
}%
{%
\caption[Works on zero- and one-dimensional modelling]
{%
	Works on zero- and one-dimensional turbocharger and engine modelling%
}%
\label{tab:works_on_turbocharging}%
}%
\end{table}

\clearpage

\begin{figure}
\ffigbox[\FBwidth]
{
\includegraphics[width=0.95\textwidth]{nested_loops.pdf}
}
{
\caption[Simulink Model with two Algebraic Loops]%
{%
	The \smlnk{} model containing the core reservoirs model and the diffuser, forming two nested \glspl{al}.
	This is the actual \smlnk{} model according to the scheme shown in \cref{fig:nested_loops}%
}%
\label{fig:nested_loops_sim}
}
\end{figure}

\clearpage

\section{Code}

\begin{lstlisting}[caption={[Vaneless adiabatic diffuser]Vaneless adiabatic diffuser, \iecfeg{cf.}\ \cref{ch:diffuser}},label={matlab:stanitz}]
function [Pi_D, M_3, Theta_D, alpha_3] = ...
	diffuser(M_2, alpha_2, r_d, r_o, h_d, k_df, adexp)
%{
Assumptions here: diffuser is adiabatic, therefore (since there is also no
work input) the stagnation temperature remains constant across the radius
and its derivate is zero, therefore neglected. The other derivate we do not
care for is that of diffuser channel height. It is simply assumed constant,
its derivate therefore zero. Adexp is the normal adiabatic exponent
'kappa'; zeta is the (empirical) friction coefficient. All angles in RAD.
The solver (ode45 as a normal standard) requires a (time)-span to work
with; just taking the 'time' to mean 'radius' does not make a difference.
The span starts at 1 (diffuser inlet: r_o/r_o = 1) and ends at r_d/r_o,
i.e. diffuser exit. Also provide initial conditions; note the substitution:
an initial value for Gamma, not alpha* is provided.
%}
R_span = [1 r_d/r_o];
R_0 = [M_2^2 tan(alpha_2)];
[R, y] = ode45(@(R,y) diffuser_flow(R, y, adexp, k_df, r_o, h_d),...
	R_span, R_0);

M = sqrt(y(:,1));
M_3 = M(end);

alpha = atan(y(:,2));
alpha_3 = alpha(end);

Pi = (r_o/r_d) .* (cos(alpha_2) ./ cos(alpha)) .* ...
	(M_2 ./ M) .* sqrt( (1 + M_2^2*(adexp - 1)/2) ./ ...
	(1 + M.^2*(adexp-1)/2));
Pi_D = Pi(end);

Theta = (1 + (adexp - 1)/2 * M_2^2)./(1 + (adexp - 1)/2 .* M.^2);
Theta_D = Theta(end);
end
function dydR = diffuser_flow(R, y, adexp, k_df, r_o, h_d)
%{
M^2 is y(1), tan(alpha) is y(2). We subsitute tan(alpha) = Gamma, so that
y(2) = tan(alpha) = Gamma and alpha = atan(y(2)).
%}
	dydR = zeros(2, 1);
	zeta = k_df * r_o / h_d;
	dydR(1) = ...
		-2*y(1) * (1 + (adexp -1)/2*y(1) )/(y(1) - ...
		sec(atan(y(2)))^2) *...
		((adexp * y(1) - y(2)^2 )*zeta/(cos(atan(y(2))))...
		- sec(atan(y(2)))^2 / R);
	dydR(2) = ...
		y(2) / (y(1) * cos(atan(y(2)))^2 -1 ) * ...
		((1 + (adexp - 1 ) * y(1)) * zeta/(cos(atan(y(2)))) - y(1) / R);
end
\end{lstlisting}

\begin{lstlisting}[caption={[Water saturation vapour pressure over temperature]Water saturation vapour pressure over temperature, \iecfeg{cf.}\ \cref{ch:humid_air}},label={matlab:buck}]
T_BuckData = linspace(273.15,373.15,200);

p_Buck = Buck(T_BuckData);

T_Buck = Simulink.Breakpoint;
	T_Buck.Breakpoints.Unit = 'K';
	T_Buck.Breakpoints.Description = ['Temperature Breakpoint Data',...
	' for Water Vapour Checking'];
	T_Buck.Breakpoints.Value = T_BuckData;

p_H2Osat = Simulink.LookupTable;
	p_H2Osat.Table.Value = p_Buck;
	p_H2Osat.Table.Unit = 'Pa';
	p_H2Osat.StructTypeInfo.Name = 'LUT_Buck';
	p_H2Osat.Breakpoints = {'T_Buck'};
	p_H2Osat.Table.Description = ['Equilibrium Water Vapour',...
	' Pressure over temperature'];

clear('p_Buck', 'T_BuckData');

function p = Buck(T)
% Returns vector of water vapour saturation pressure over a temperature
% vector. Temperature in Kelvin, Pressure in Pascal.
	p = 611.21.*exp( (18.678-((T - 273.15)./234.5)).*...
		((T - 273.15)./(257.14+(T - 273.15))) );
end
\end{lstlisting}

\begin{lstlisting}[caption={[Geometry Inference Tool] Geometry inference tool, \iecfeg{cf.}\ \cref{ch:geometric_similitude}. To generate Tables as used in this work, refer to \cref{matlab:geometry_tables}},label={matlab:geometry_inferrer}]
function geometry_inferrer
% Appends stats to an existing table. From the stats and user input, infers
% all missing values. The result is printed and the stats table saved as
% *.MAT in the same DIR the function is called from.

%% Set expected Defaults and other auxiliaries
%{
These are only used as suggestions and pre-filled into the dialogue. The
user can change these anytime.
%}
dflt.comp = 'C';
dflt.turb = 'T';
dflt.filename.data = 'machine_parameters.m';
dflt.filename.stats = 'geostats.mat';
btnFF = 'From File';
btnWS = 'From Base Workspace';
btnNo = 'No';
fullfilepath = mfilename('fullpath');
[filepath, filename, ~] = fileparts(fullfilepath);
%% Begin Dialogue
%{
Extract Data from user-specified input. Can be either a File that is run
(an *.m-file), variables in the Base Workspace or existing data from a
previous run (a *.MAT-file). All variables are expected to be in Table
format, which is the data type best suited for this work. Therefore, we
force it. No funny business with dynamically named variables or naked
matrices allowed.
%}
pass_data = questdlg({'Would you like to pass existing machine data?', ...
    'Its data would be used to compute your request.', ...
    ['Regardless, note that data is expected to be in ',...
    'Tables named ''', dflt.comp, ''' and ''', dflt.turb, '''.'], '',...
    'If you choose no, existing values are used.'}, ...
    'Machine Data Prompt', btnFF, btnWS, btnNo, btnFF);
switch pass_data
    case btnFF
        prompt = {'File name containing the Tables:', ...
            'Compressor Table name in that file:',...
            'Turbine Table name in that file:'};
        title = 'Machine Data Input';
        dims = [1 50];
        definput = {dflt.filename.data, dflt.comp, dflt.turb};
        machine_data_file = inputdlg(prompt, title, dims, definput);
        if isempty(machine_data_file)
            fprintf('[%s] You left the dialogue and function.\n',...
                datestr(now));
            return
        end
        run(machine_data_file{1});%spills file contents into funcWS
        C = eval(machine_data_file{2});
        T = eval(machine_data_file{3});
    case btnWS
        prompt = {'Base Workspace Compressor Table name:',...
                  'Base Workspace Turbine Table name:'};
        title = 'Machine Data Input';
        dims = [1 50];
        definput = {dflt.comp, dflt.turb};
        machine_data_ws = inputdlg(prompt, title, dims, definput);
        if isempty(machine_data_ws)
            fprintf('[%s] You left the dialogue and function.\n',...
                datestr(now));
            return
        end
        C = evalin('base', machine_data_ws{1});
        T = evalin('base', machine_data_ws{2});
    case btnNo
        boxNo = msgbox(['Looking for stats in ''', dflt.filename.stats, ...
            '''.'], 'Using Existing Data', 'help');
        waitfor(boxNo);
        try
            stats = load(dflt.filename.stats);
            stats = stats.stats;
        catch ME
            switch ME.identifier
                case 'MATLAB:load:couldNotReadFile'
                    warning(['File ''', dflt.filename.stats, ''' not ',...
                        'found in search path. Make sure it has been ',...
                        'generated in a previous run or created ',...
                        'manually. Resorting to hard-coded data ',...
                        'for now.']); `\phnote{see \cref{tab:statistical_parameters}}`
                        R.ihC = 3.0045;
                        R.oiC = 1.52;
                        R.doC = 1.54;
                        beta_1geoC = 138.4;
                        R.ihT = 2.71;
                        R.oiT = 1.17;
                        H_T = 3.512;
                        beta_2geoT = 139.1;
                otherwise
                    rethrow(ME);
            end
        end
    case ''
        fprintf('[%s] You left the dialogue and function.\n',datestr(now));
        return
    otherwise%Only gets here if buttons are misconfigured
        error('This option is not coded, should not get here.');
end
name.mean = 'MEAN';
name.stdev = 'STDEV';
name.coeffvar = 'CV''';
%% Compute Stats
%{
If user specified input of newly available data, use it here to compute a
new stats file.
%}
if strcmp(pass_data, btnFF) || strcmp(pass_data, btnWS)
    validTab = @(x,y) validateattributes(x, {'table'}, {'nonempty'},...
        filename, y);
    validTab(C, 'Compressor Data');
    validTab(T, 'Turbine Data');
    C = delusrdf(C);
    T = delusrdf(T);    
    [trbstats, cmprstats] = deal(table);    
    cmprstats.R_ih = tblbld(C.r_i./C.r_h);
    cmprstats.R_oi = tblbld(C.r_o./C.r_i);
    cmprstats.H_o = tblbld(C.r_o./C.h_d);
    cmprstats.R_do = tblbld(C.r_d./C.r_o);
    cmprstats.beta_1geo = tblbld(C.beta_1geo);    
    trbstats.R_ih = tblbld(T.r_i./T.r_h);
    trbstats.R_oi = tblbld(T.r_o./T.r_i);
    trbstats.H_o = tblbld(T.r_o./T.h);
    trbstats.beta_2geo = tblbld(T.beta_2geo);    
    assert(isequal(C.Properties.RowNames, T.Properties.RowNames), ...
        'Compressor and Turbine Table Row Names differ.');
    names = C.Properties.RowNames;
    names{end + 1} = name.mean;
    names{end + 1} = name.stdev;
    names{end + 1} = name.coeffvar;    
    [cmprstats.Properties.RowNames,trbstats.Properties.RowNames] = ...
        deal(names);    
    stats = table(cmprstats, trbstats);    
    stats.Properties.RowNames = names;    
    stats.Properties.Description =['Turbo-Machinery Statistical ',...
        'Parameters. Filtered values are NaN.'];
    fprintf('[%s] Using stats computed from input.\n', datestr(now));    
    save(fullfile(filepath, dflt.filename.stats), 'stats');
elseif strcmp(pass_data, btnNo)
    fprintf('[%s] Using existing stats.\n', datestr(now));
else%Only gets here if buttons are misconfigured
    error('This option is not coded, should not get here.');
end
%% Compute Approximations from User Input
%{
Only run if stats file is present; otherwise, hard-coded data is
overwritten if user did not supply anything.
%}
if exist('stats', 'var') == 1
    R.ihC = stats.cmprstats{name.mean, 'R_ih'};
    R.oiC = stats.cmprstats{name.mean, 'R_oi'};
    R.doC = stats.cmprstats{name.mean, 'R_do'};
    R.ihT = stats.trbstats{name.mean, 'R_ih'};
    R.oiT = stats.trbstats{name.mean, 'R_oi'};
    H_T = stats.trbstats{name.mean, 'H_o'};
    beta_1geoC = stats.cmprstats{name.mean, 'beta_1geo'};
    beta_2geoT = stats.trbstats{name.mean, 'beta_2geo'};
end
varnames.r_h = 'r_h';
varnames.r_i = 'r_i';
varnames.r_o = 'r_o';
varnames.r_d = 'r_d';
varnames.h = 'h';
syms(struct2cell(varnames));
prompt = {['Compressor ', varnames.r_h, ':'],...
          ['Compressor ', varnames.r_i, ':'],...
          ['Compressor ', varnames.r_o, ':'],...
          ['Compressor ', varnames.r_d, ':'],...
          ['Turbine ', varnames.r_h, ':'],...
          ['Turbine ', varnames.r_i, ':'],...
          ['Turbine ', varnames.r_o, ':'],...
          ['Turbine ', varnames.h, ':']};
title = 'Known Parameter Input';
dims = [1 40];
[definput{1:length(prompt)}] = deal('-');
opts.Interpreter = 'tex';
param_inpt = inputdlg(prompt, title, dims, definput, opts);
param_inpt = num2cell(str2double(param_inpt));
[Cg.(varnames.r_h), Cg.(varnames.r_i), Cg.(varnames.r_o), ...
    Cg.(varnames.r_d), Tg.(varnames.r_h), Tg.(varnames.r_i), ...
    Tg.(varnames.r_o), Tg.(varnames.h)] = ...
    param_inpt{:};
eqn.C1 = r_i/r_h == R.ihC;
eqn.C2 = r_o/r_i == R.oiC;
eqn.C3 = r_d/r_o == R.doC;%Compressor only
eqn.T1 = r_i/r_h == R.ihT;
eqn.T2 = r_o/r_i == R.oiT;
eqn.T3 = r_o/h == H_T;%Turbine only
% Order of Symbolics has to be the same as in the geoStruct that specifies
% them (look into Cg and Tg for reference)
Compressor = solveGeo(Cg, [eqn.C1 eqn.C2 eqn.C3], [r_h r_i r_o r_d]);
Turbine = solveGeo(Tg, [eqn.T1 eqn.T2 eqn.T3], [r_h r_i r_o h]);
Compressor(end + 1, 1) = {beta_1geoC};
Compressor.Properties.RowNames(end) = {'beta_1'};
Turbine(end + 1, 1) = {beta_2geoT};
Turbine.Properties.RowNames(end) = {'beta_2'};
approxs = table(Compressor, Turbine);
approxs.Properties.Description = ['Approximate dimensions for a ',...
    'radial compressor and radial turbine. For angles, only the mean ',...
    'is specified and no inference took place.'];
%Print computed approximations (no semicolon) and append summary
approxs.Properties.DimensionNames ={'GeometryParameters', 'MachineType'}
summary(approxs)
end
function tblObj = delusrdf(tblObj)
%See if there is a table row specified for User Defined input. This could
%be used for Debugging the simulation, but could ruin the stats.
    try
        usrrow = 'User Defined';
        tblObj({usrrow},:) = [];
        fprintf(['[%s] ''', usrrow, ''' row found & deleted for ',...
            'analysis.\n'], datestr(now));
    catch
        fprintf(['[%s] ''', usrrow, ''' *not* row found & deleted for ',...
            'analysis.\n'], datestr(now));
    end
end
function S = solveGeo(geo_struct, eqns_in, symbs_in)
% Take in a struct of geometry data, the equations that apply and all
% symbolic variables that occur in these. Solve the eqns and store in a
% Table column. The order of symbolics is highly important here.
fln = fieldnames(geo_struct);
S = table(NaN(length(fln),1), 'RowNames', ...
        fln, 'VariableNames', {'approx'});
switch nnz(~structfun(@isnan, geo_struct))
    case 0
        fprintf('[%s] Looks like no action is wanted.\n', datestr(now));
    case 1
        fprintf('[%s] Acting from one parameter.\n', datestr(now));
        pick = fln(~structfun(@isnan, geo_struct));
        eqn_pick = evalin('caller',pick{1}) == geo_struct.(pick{1});`\label{line:geo_system_of_eqns}`
        eqns = [eqns_in eqn_pick];
        S.approx = structfun(@double, solve(eqns, symbs_in));
    case 2
        fprintf('[%s] Please enter only one parameter.\n', datestr(now));
    otherwise
        fprintf('[%s] Looks like no action is required.\n', datestr(now));
end
end
function tblelement = tblbld(vec)
%Constructs one table column
    tblelement = statsvec(vec.*~outl(vec));
    function stdvec = statsvec(vec)
    %Returns Vector with Statistics appened at end. Does not work if data
    %contains zeroes, these are turned to NaN here so that they do not
    %contribute to stats computation.
       vec(vec == 0) = NaN;
       stdvec = [vec; mn(vec); stdev(vec);...
                (1 + 1/(4*length(vec(~isnan(vec)))))*stdev(vec)/mn(vec)];
    end
end
function mean = mn(vec)
%Returns mean of vector
    vec = vec(~isnan(vec));
    mean = sum(vec)/length(vec);
end
function std = stdev(vec)
%Returns Standard Deviation of the Sample of Vector
    vec = vec(~isnan(vec));
    std = sqrt((sum((vec - mn(vec)).^2))/(length(vec) - 1));
end
function median = med(vec)
 %Returns median of vector
    vec = sort(vec);
    if length(vec) == 1
        median = vec;
    elseif mod(length(vec),2) == 0
        median = (vec(length(vec)/2) + vec(length(vec)/2 + 1))/2;
    else
        median = vec(ceil(length(vec)/2));
    end
end
function outliers = outl(vec) `\phnote{see \cref{eq:stats_MAD}}`
%Returns outliers within vector based on Median Absolute Deviation
    MAD = med(abs(vec - med(vec)));
    scaling = 4.5;
    outliers = (abs(vec - med(vec)) >= scaling * MAD);
    if any(outliers)
        fprintf('[%s] %d of %d were found to be outliers.\n',...
            datestr(now), nnz(outliers), length(outliers));
    else
        fprintf('[%s] No outliers detected.\n', datestr(now));
    end
end
\end{lstlisting}

\ifpublished
	\begin{lstlisting}[caption={[Geometry parameter tables]Tables holding geometry parameters, \iecfeg{cf.}\ \cref{ch:geometric_similitude,tab:turbine_geometry,tab:compressor_geometry}}, label={matlab:geometry_tables}]
		Content not printed in published version.
	\end{lstlisting}
\else
\begin{lstlisting}[caption={[Geometry parameter tables]Tables holding geometry parameters, \iecfeg{cf.}\ \cref{ch:geometric_similitude,tab:turbine_geometry,tab:compressor_geometry}}, label={matlab:geometry_tables}]
%% Machine Data Parameterization
%{
The first row is reserved for arbitrary user input, so that the remaining
machine data/model is not dirtied. Use TC_Name (current turbocharger name)
and E_Name to set the current machine to be used! NaN entries mean that no
data was available at time of writing.
%}
tc_names = {'User Defined';'VATN'; 'T03'; 'K26'; 'ZR1'; 'ZR3'; 'ABB'};            
TC_Name = tc_names{5};
[C, T, S] = deal(table('RowNames',tc_names));
%% Table: Compressors
C.r_h = [0.00675; 0.0111; 0.00775; 0.00675; 0.013; 0.021; 0.0236];
C.r_i = [0.0210; 0.02115; 0.02025; 0.0210; 0.04325; 0.065; 0.067883];
C.r_o = [0.033; 0.031; 0.030; 0.033; 0.0675; 0.1025; 0.10];
C.r_d = [0.07; 0.05; 0.05; 0.07; 0.103; 0.1625; 0.13238];
C.h_d = [0.0072; 0.006325; 0.0046; 0.0072; 0.0074; 0.0112; 0.01156];
%Careful: Angles are converted from Lyssewski to Povel
C.beta_1geo = [142.5; 143.5; 143.5; 142.5; 136.2; 135.4; 129];
C.beta_2geo = [120; 120; 120; 120; 150; 150; 150];
C.Z_1 = [NaN; NaN; NaN; NaN; 6; 8; 8];
C.Z_2 = [12; 18; 12; 12; 12; 16; 16];
C.s = [0.001; 0.001; 0.001; 0.001; 0.001; 0.001; 0.001];%all guesstimated
C.b_1 = [NaN; NaN; NaN; NaN; 0.0012; 0.0013; 0.0013];%blade thickness in
C.b_2 = [NaN; NaN; NaN; NaN; 0.0014; 0.0024; 0.0024];%blade thickness out
C.Properties.Description = 'Compressor Parameters';
C.Properties.VariableDescriptions = {'Hub Radius',...
    'Inner Blade Radius (Inlet)', 'Outer Blade Radius (Outlet)',...
    'Diffuser Radius', 'Axial Length of Diffuser (''height'')',...
    'Rotor Inlet Metal Angle', 'Rotor Outlet Metal Angle',...
    'No. of Full Blades', 'Number of Full and Splitter Blades', ...
    'Clearance as a fraction of blade height',...
    'Inducer Calibrated Blockage Factor', ...
    'Discharge Calibrated Blockage Factor'};
C.Properties.VariableUnits = {'m', 'm', 'm', 'm', 'm', 'deg',...
                            'deg', '1', '1', '1', '1', '1'};
C.Properties.DimensionNames = {'Names', 'Parameters'};
%% Table: Turbines
T.r_h = [0.01075; 0.00775; 0.0084; 0.01075; 0.025; 0.029; 0.031375];
T.r_i = [0.02725; 0.0239; 0.0245; 0.02725; 0.0565; 0.0765; 0.088815];
T.r_o = [0.032; 0.03; 0.0315; 0.032; 0.0615; 0.09; 0.09259];
T.h = [0.0102; 0.0075; 0.0105; 0.0102; 0.0165; 0.023; 0.02812];
%alpha_1 can only be set if stator is vaned. otherwise, it is calculated
T.alpha_1 = [45; 45; 45; 45; NaN; NaN; 23.4];
T.beta_1geo = [90; 90; 90; 90; 90; 90; 90];
T.beta_2geo = [140; 147; 156; 140; 128; 131.7; 131.7];
T.Z = [12; 12; 11; 12; 11; 13; 12];
T.b_1 = [NaN; NaN; NaN; NaN; 1.2e-3; 1.2e-3; 1.2e-3];%blade thickness in
T.b_2 = [NaN; NaN; NaN; NaN; 3.1e-3; 3.4e-3; 3.4e-3];%blade thickness out
T.AR = [NaN; NaN; NaN; NaN; 0.0285; 0.0420; 0];
T.vaned_stator = [NaN; NaN; NaN; NaN; false; false; true];
T.Properties.Description = 'Turbine Parameters';
T.Properties.VariableDescriptions = {'Hub Radius',...
    'Blade Radius (Outlet)', 'Blade Radius (Inlet)',...
    'Axial Stator Length (''height'')', 'Stator Outlet Metal Angle',...
    'Rotor Inlet Metal Angle', 'Rotor Outlet Metal Angle',...
    'No. of Blades', 'Inlet Calibrated Blockage Factor', ...
    'Outler Calibrated Blockage Factor', 'Vaneless Stator Factor',...
    'Possesion of vaned stator'};
T.Properties.VariableUnits = {'m', 'm', 'm', 'm', 'deg', 'deg',...
                            'deg', '1', '1', '1', '1', 'boolean'};
T.Properties.DimensionNames = {'Names', 'Parameters'};
%% Table: Shaft
S.n_R = [108e3; 108e3; 108e3; 108e3; 80e3; NaN; NaN]/60;%all guessed
S.I = [0.000136; NaN; 0.00012; 0.000136; 0.00239; 0.01346; 0.02442];
S.Properties.Description = 'Turbocharger Shaft Parameters';
S.Properties.VariableDescriptions = {'Rated (max) Rotational Speed',...
                    'Moment of Inertia'};
S.Properties.VariableUnits = {'1/s','kg*m^2'};
S.Properties.DimensionNames = {'Names', 'Parameters'};
%% Table: Engines
eng_names = {'User Defined'; 'MaK 9M43 C'; 'Daimler-Benz OM 617A'};
E_Name = eng_names{2};
E = table('RowNames',eng_names);
E.V_CD = [0.002998; 0.0886; 0.0006];
E.z_cyl = [5; 9; 5];
E.P_cyl = [NaN; 900000; NaN];
E.b_e = [NaN; 4.91667e-8; NaN];
E.I = [5; 4760; 5];
E.p_RCA = [172000; 310000; 1720000];
E.T_RCA = [318.15; 318.15; 318.15];
E.n_R = [70; 8.333; 70];
E.a = [2; 2; 2];
E.V_exh = [0.01; 6.48; 0.01];
E.V_int = E.V_exh;
E.Properties.Description = 'IC-Engine Parameters';
E.Properties.VariableDescriptions = {'Cylinder Displacement',...
                    'Number of Cylinders', 'Rated Power per Cylinder',...
                    'Specific Fuel Oil Consumption',...
                    'Polar Moment of the Crankshaft with Attachments',...
                    'Rated Charge Air Pressure',...
                    'Rated Charge Air Temperature',...
                    'Rated Rotational Speed', 'Working Principle',...
                    'Exhaust Manifold Volume', 'Inlet Manifold Volume'};
E.Properties.VariableUnits = {'m^3','1', 'W', 'kg/J', 'kg*m^2','Pa',...
    'K','1/s','1','m^3','m^3'};
E.Properties.DimensionNames = {'Names', 'Parameters'};
%% Clean-Up
clear('tc_names', 'eng_names');
\end{lstlisting}
\fi

\clearpage%MANUAL clearing
\begin{lstlisting}[caption = {[Test-stand function]Test-stand function, simulating a compressor or turbine on an actual, experimental test-stand as described in \cref{il:hot_gas,ch:test-stand}. For each rotational speed to be tested, one simulation is run in parallel. The function is mainly concerned with supplying and extracting data}, label={matlab:teststand}]
function simOut = teststand(varargin)
%Returns Pi-over-dotm etc. lines of constant speed for turbomachines.
%{
Script for Turbocharger Performance Map Generation. On a hot-gas test
stand, in order to meassure the compressor map, a valve behind the machine
is closed continuously. The test begins with a fully opened valve behind
the machine, such that very low back-pressure is induced and only a small
pressure ratio is built up. From there, the valve starts closing, building
up back-pressure over time. As a consequence, the pressure rises, but mass
throughput decreases. At or near the surge line, the test stops. 'p' is the
pressure behind the machine, representing the valve closing (in Pa) for the
compressor. For a turbine, 'p' is the pressure in front. This script tries
to reenact a real test stand, where the testing is also a process over
time. First, measure execution time for diagnostics:
%}
simInfo.tau.start.('script') = tic;
%% Parse Input
%{
Validate that the input fulfills certain criteria. Also declare required
and optional inputs, as well as optional parameters. The required and
optional inputs are positional; that means, that the order *must* be as
they occur when they are 'add'ed here. Conversely, 'addParameter'
parameters can occur in any order. Example: "simulation_Output =
simulation_Function('model_name', number_of_scenarios, 'max_step',1e-2)".
The 'inputParser' object is then parsed; if required inputs are missing, an
error is thrown. If criteria aren't met, an error is thrown. If unknown
parameters are requested, an error is thrown. The goal is of course to stop
funny business like imaginary numbers. We don't set the rotational speed
defaults here, they are set later with the help of the Data Dictionary. The
target is the simulation's target model, which can be from LYS (Lyssewski)
or POV (Povel); these work quite differently, therefore different treatment
is required.
%}
default.model_name = 'compressor_radial_StaticCode';
default.no_of_scenarios = 3;
default.max_step = '1e-1';
%We do not actually want to use these, they are set later. Just check
%whether defaults are being used or not. If anything slips, putting NaN
%here helps tracking the bug from the resulting error.
default.n_min = NaN;
default.n_max = NaN;

default.plot = true;
default.author = 'LYS';
default.scale = 1;
default.write = true;
default.redplot = true;

P = inputParser;
P.CaseSensitive = false;
P.FunctionName = 'teststand';

validNumber = @(x) validateattributes(x,...
    {'numeric'},{'positive','nonempty'});
validScenarioNumber = @(x) validateattributes(x,...
    {'numeric'},{'integer','nonempty','>',1,'scalar'});
validChar = @(x) validateattributes(x,{'char'},{'nonempty'});
validBoolean = @(x) validateattributes(x,{'logical'},{'nonempty'});
validShare = @(x) validateattributes(x,...
    {'numeric'},{'positive','nonempty','>',0,'<',1}); 

addOptional(P,'no_of_scenarios',default.no_of_scenarios,...
    validScenarioNumber);
addOptional(P,'model_name',default.model_name,validChar);

addParameter(P,'author',default.author,validChar);
addParameter(P,'n_min',default.n_min,validNumber);
addParameter(P,'n_max',default.n_max,validNumber);
addParameter(P,'max_step',default.max_step,validNumber);
addParameter(P,'plot',default.plot,validBoolean);
addParameter(P,'scale',default.scale,validShare);
addParameter(P,'write',default.write,validBoolean);
addParameter(P,'redplot',default.redplot,validBoolean);

parse(P,varargin{:});

%% Find and Load Model representing the Test Stand
%{
Create Struct to hold all names and info for all kinds of different
purposes.
%}
simInfo.names.('model') = P.Results.model_name;
%{
The exist() function returns '4' if input exists and is a Simulink Model.
%}
assert(exist(simInfo.names.('model'),'file') == 4,...
    'No Model of name ''%s'' found.',simInfo.names.('model'));
%{
Before commencing, check if the Simulink Model that is being loaded has in
its name the type of machine. Depending on that name, the script decides
whether to run a compressor or turbine performance simulation. If file is
not named properly, notify. Afterwards, set a logical that determines
whether the machine at hand is a compressor (=true/1). If untrue, turbine
is implied. Note that the pattern against which we check have to be strings
(double quote ") instead of just character arrays (single '). Since
Compressor OR Turbine is a binary decision, a Boolean is a good tool (as
opposed to switch/case).
%}
isCompressor = contains(simInfo.names.('model'), "compressor", ...
    'IgnoreCase', true);
if isCompressor
    simInfo.names.('machine_type') = 'Compressor';
else
    if contains(simInfo.names.('model'), "turbine",'IgnoreCase', true)
        simInfo.names.('machine_type') = 'Turbine';
    else
        isCompressor = true;
        simInfo.names.('machine_type') = 'Compressor';        
        warning(['The model to be run contains neither ''compressor'' ',...
        'nor ''turbine'' in its name, as would be required for ',...
        'automatic distinction. Setting it to ''%s'' as default.'],...
        simInfo.names.('machine_type'));
    end
end
fprintf(['[%s] It was inferred that ',...
            'this machine is a %s.\n'], datestr(now),...
            lower(simInfo.names.('machine_type')));
%{
Load the Model in order to be able to Compile and modify it. Using
'open_system' actually opens a window with the Simulink Model.
'load_system' loads it into memory in the background. If the model is
already open, nothing further happens.
%}
load_system(simInfo.names.('model'));
%{
Forward the names of the Input Ports of the Rotational Speed input,
backpressure input etc. If they are changed in the model, change them here
too. Putting them into sub-struct for better overview. These are
Case-Sensitive, take care.
%}
switch P.Results.author
    case 'POV'
        simInfo.names.ports.in.('rot_speed') = 'n';
        simInfo.names.ports.in.('temperature') = 'T_0';
        simInfo.names.ports.out.('pressure_ratio') = 'Pi';
        simInfo.names.ports.out.('enthalpy_stream') = 'Stream';
        if isCompressor
            simInfo.names.ports.in.pressure.('exit') = 'p';
            simInfo.names.ports.in.pressure.('entry') = 'p_0';
        else
            simInfo.names.ports.in.pressure.('exit') = 'p_0';
            simInfo.names.ports.in.pressure.('entry') = 'p';
        end
    case 'LYS'
        simInfo.names.ports.in.('in_state') = 'in_state';
        simInfo.names.ports.in.pressure.('back') = 'p_back';
        simInfo.names.ports.in.('rot_speed') = 'n';
        simInfo.names.ports.in.('scale') = 'scale';
        simInfo.names.ports.out.('enthalpy_stream') = 'stream';
        simInfo.names.ports.out.('pressure_ratio') = 'Pi';
        simInfo.names.ports.out.('efficiency') = 'eta';
        simInfo.names.ports.out.('torque') = 'M';
    otherwise
        error('No model author codenamed ''%s'' known.', P.Results.author);
end
%% Find and Load Data Dictionary
%{
Use inital pressure offset and other parameters from Data Dictionary. We
use the parent dictionary, since it already references the 'parameter'
dictionary and all others. Look for it, so it can still be found after
moving. Name of the expected Data Dictionary is put in. Afterwards, add
Suffix so it is not confused with accompanying *.m-file.
%}
simInfo.names.('dd_parent') = 'parent';
simInfo.names.('dd_parent') = strcat(simInfo.names.('dd_parent'),'.sldd');

simInfo.names.('dd_maps') = 'performance_maps';
simInfo.names.('dd_maps') = strcat(simInfo.names.('dd_maps'),'.sldd');
%{
The exist() function returns '2' if input exists and is a regular file. The
assert() function asserts a logical expression, and complains with an error
if the assertion fails/logical returns false.
%}
assert(exist(simInfo.names.('dd_parent'),'file') == 2,...
'No Data Dictionary of name ''%s'' found.',simInfo.names.('dd_parent'));

assert(exist(simInfo.names.('dd_maps'),'file') == 2,...
'No Data Dictionary of name ''%s'' found.',simInfo.names.('dd_maps'));
%{
Load the DD ('represent' it) into the Function Workspace using a Variable.
The function which() returns full path. Load the Data Dictionary's Section
'Design Data' afterwards, also represent it.
%}
dObj = Simulink.data.dictionary.open(which(simInfo.names.('dd_parent')));
dSectObj = getSection(dObj,'Design Data');
%% Load Required Parameters from DD
%{
Having loaded the correct Section (there is just one for this, 'Design
Data'), we start grabbing its Entries. An Entry is only an object that
contains the value we are looking for, but also a lot of metadata info
(Last Changed by) etc., which we don't need. So after the Entry is grabbed,
grab its Value. This Value can for example be a SimulinkParameter or a
struct. The former contains additional info too, like units. Its actual
numerical value is retrieved as needed with 'dValueObj.Value'. We also
require additional info like ambient conditions, as well as the currently
selected machine model we are working on. This is used e.g. in the plot
legend.
%}
dEntryObj = getEntry(dSectObj,'T_inletT');
dValueObj = getValue(dEntryObj);
T_inletT = dValueObj.Value;

dEntryObj = getEntry(dSectObj,'ambient_no');
dValueObj = getValue(dEntryObj);
simInfo.number_of.('ambient') = dValueObj.Value;

dEntryObj = getEntry(dSectObj,'ambient');
dValueObj = getValue(dEntryObj);
simInfo.('ambient_conditions') = dValueObj(simInfo.number_of.('ambient'));

dEntryObj = getEntry(dSectObj,'air_no');
dValueObj = getValue(dEntryObj);
simInfo.number_of.('air') = dValueObj.Value;

dEntryObj = getEntry(dSectObj,'air');
dValueObj = getValue(dEntryObj);
simInfo.('air') = dValueObj(simInfo.number_of.('air'));

dEntryObj = getEntry(dSectObj,'exh_no');
dValueObj = getValue(dEntryObj);
simInfo.number_of.('exh') = dValueObj.Value;

dEntryObj = getEntry(dSectObj,'exh');
dValueObj = getValue(dEntryObj);
simInfo.('exh') = dValueObj(simInfo.number_of.('exh'));
%% TimeSeries: Pressure
t = [0; 200];
%{
Initial Pressure is higher than Ambient by some margin. Final Pressure
is a certain multiple of Initial (i.e. ambient) pressure. This is only to
cover a wide range of possible pressure ratios. The simulation will likely
stop before its designed stop time and before the high pressure ratios are
reached anyway, once surging/zero mass flux occurs. Linearly Growing Vector
of same length as time vector. Notice small apostrophe to transpose vector.
%}
p = linspace(... `\label{line:teststand_p}`
    1.1 * simInfo.('ambient_conditions').p,...
    5 * simInfo.('ambient_conditions').p,...
    length(t));

p = timeseries(p',t,'name','back_pressure');
p.DataInfo.Units = 'Pa';
%% TimeSeries: Constants
%{
Simply constant timeseries for the ambient conditions at their respective
ports, among others.
%}
T_amb = timeseries(simInfo.('ambient_conditions').T,t,...
    'name','ambient_temperature');
p_amb = timeseries(simInfo.('ambient_conditions').p,t,...
    'name','ambient_pressure');

if isCompressor
    phi.O2 = timeseries(simInfo.('air').phi_O2, t, ...
        'name','inlet air oxygen volumetric concentration');
    phi.N2 = timeseries(simInfo.('air').phi_N2, t, ...
        'name','inlet air nitrogen volumetric concentration');
    phi.Ar = timeseries(simInfo.('air').phi_Ar, t, ...
        'name','inlet air argon volumetric concentration');
    phi.CO2 = timeseries(simInfo.('air').phi_CO2, t, ...
        'name','inlet air carbon dioxide volumetric concentration');
    phi.H2Og = timeseries(0, t, ...
        'name','inlet air water vapour volumetric concentration');
    phi.SO2 = timeseries(0, t, ...
        'name','inlet air sulphur dioxide volumetric concentration');
else
    phi.O2 = timeseries(simInfo.('exh').phi_O2, t, ...
        'name','exhaust gas oxygen volumetric concentration');
    phi.N2 = timeseries(simInfo.('exh').phi_N2, t, ...
        'name','exhaust gas nitrogen volumetric concentration');
    phi.Ar = timeseries(simInfo.('exh').phi_Ar, t, ...
        'name','exhaust gas argon volumetric concentration');
    phi.CO2 = timeseries(simInfo.('exh').phi_CO2, t, ...
        'name','exhaust gas carbon dioxide volumetric concentration');
    phi.H2Og = timeseries(simInfo.('exh').phi_H2O, t, ...
        'name','exhaut gas water vapour volumetric concentration');
    phi.SO2 = timeseries(simInfo.('exh').phi_SO2, t, ...
        'name','exhaust gas sulphur dioxide volumetric concentration');    
end
scale = timeseries(P.Results.scale, t, ...
    'name', 'scaling factor for the turbomachine (no scaling: 1)');
%% TimeSeries: Rotational Speed
%{
Amount of Rotational Speed Values to be tested between min and max. This
is equivalent to the amount of simulations that need to be run.
%}
simInfo.number_of.('scenarios') = P.Results.no_of_scenarios;
%{
Rated (maximum) Turbocharger Rotational Speed (TCR), after which testing
halts. This info has moved to an *.m-file (used to be in the Data
Dictionary, but DDs do not support 'table' objects). From the *.m-file with
the machine parameters, the Model Workspace is initiated. Thus, get the
Model Workspace of the model we are testing here. Then, extract the name of
the turbocharger we are currently testing. If you want to change this,
change it in the machine parameters' *.m-file! Afterwards, get table that
contains the turbochargers rated rotational speed. Use it to set the upper
bound of rotational speeds to be tested. In the *.m-file, the names of the
'table' objects are as follows: C for Compressor, T for Turbine, S for
shaft (of the turbocharger. This is used for values such as max. rot.
speed, inertia etc.) and lastly E for engine.
%}
mdlWks = get_param(simInfo.names.('model'),'ModelWorkspace');
reload(mdlWks);
simInfo.names.('turbocharger') = getVariable(mdlWks,'TC_Name');
S = getVariable(mdlWks,'S');
n_TCR = S.n_R(simInfo.names.('turbocharger'));
%{
Initial Rotational Speed Fraction of the Rated Speed, yielding the initial
rotational speed:
%}
dEntryObj = getEntry(dSectObj,'init_rot_speed_frac');
dValueObj = getValue(dEntryObj);
n_0 = dValueObj.Value * n_TCR;
%{
Create Linearly Spaced vector and also convert to RPM in order to round
the RPM values. If Defaults are used, set the vector to the used grabbed
from the Data sources (Data Dictionary, Model Workspace, etc.). If any of
the two, or both, are set by the user, use these.
%}
if any(contains(P.UsingDefaults,'n_max'))
   n_max = n_TCR;
else
   n_max = P.Results.n_max;
end
if any(contains(P.UsingDefaults,'n_min'))
   n_min = n_0;
else
   n_min = P.Results.n_min;
end
if n_max > 6000
   warning(['It is very likely n_max = %d was entered in RPM. ',...
           'Please only use 1/s.'], n_max);
elseif n_min > 3000
   warning(['It is very likely n_min = %d was entered in RPM. ',...
           'Please only use 1/s.'], n_min);
end
n = linspace(n_min, n_max, simInfo.number_of.('scenarios'));`\label{line:teststand_n}`
%{
Round numbers to closest multiple of something and convert back to RPS.
Negative Value in round() denotes rounding to the left of the decimal
point. We want smooth plotting in RPM, but all calculations are done in SI
units and therefore RPS. Conversion to RPM comes last, and is only visual
for plotting.
%}
n = round(n*60,-3)/60;
if n(1) == 0
    warning(['Lowest Rotational Speed to be tested was so low, ',...
    'it was rounded to 0. Set back to 100rps/6000rpm for now. ',...
    'Consider using higher speeds.']);
     n(1) = 100;
end
%Check correctness and feed-back to user.
fprintf('[%s] Working on these rot. speeds [krpm]:\n[%s] %s\n',...
                datestr(now), datestr(now), sprintf('%d ', n*60/1000));
%% DataSet Creation
%{
Number of Expected Inputs. Currently, this number is 4: the rotational
speed and the pressure behind (compressor) or in front (turbine) of the
machine need to be specified. Further, the Input temperature is controlled
here, so is the entry (compressor) or exit (turbine) pressure; both these
are ambient properties.
%}
simInfo.number_of.('inputs') = 4;
%{
Create/Get Inport Dataset for this Model, automatically fetched from all
available inports in the Root level.
%}
inport_ds = createInputDataset(simInfo.names.('model'));

if inport_ds.numElements ~= simInfo.number_of.('inputs')
    warning(['Expected %d elements ',...
        '(one inport for ''%s'' etc.) ',...
        'in the fetched Dataset, instead got %d. ',...
        'Maybe change the amount of Inputs in the actual model. ',...
        'All superfluous Input Ports are left at their ',...
        'default of Zero!'],...
        simInfo.number_of.('inputs'),...
        simInfo.names.ports.in.('rot_speed'),...
        inport_ds.numElements);
end
%{
Get and Store indices of the input ports for 'rotational speed' as well as
the 'pressure' and others. These are important to fit the correct
timeseries to the correct Input ports.
%}
switch P.Results.author
    case 'POV'
        simInfo.indices.('rot_speed') = idxSearch(...
                    simInfo.names.ports.in.('rot_speed'),inport_ds);
        simInfo.indices.pressure.('exit') = idxSearch(...
                    simInfo.names.ports.in.pressure.('exit'),inport_ds);
        simInfo.indices.pressure.('entry') = idxSearch(...
                    simInfo.names.ports.in.pressure.('entry'),inport_ds);
        simInfo.indices.('temperature') = idxSearch(...
                    simInfo.names.ports.in.('temperature'),inport_ds);
    case 'LYS'
        simInfo.indices.('rot_speed') = idxSearch(...
                    simInfo.names.ports.in.('rot_speed'),inport_ds);
        simInfo.indices.('in_state') = idxSearch(...
                    simInfo.names.ports.in.('in_state'),inport_ds);
        simInfo.indices.('scale') = idxSearch(...
                    simInfo.names.ports.in.('scale'),inport_ds);
        simInfo.indices.pressure.('back') = idxSearch(...
                    simInfo.names.ports.in.pressure.('back'),inport_ds); 
    otherwise
        error('No model author codenamed ''%s'' known.', P.Results.author);
end                
%{
Equip the Dataset with the timeseries for the signals in the
correct slots. We can only do this for all inputs that are the same in
all scenarios, i.e. not the rotational speed. Note that for a compressor,
the input temperature is simply ambient. However, for the turbine, this is
another special parameter.
%}
switch P.Results.author
    case 'POV'        
        if isCompressor
            inport_ds = setElement(...
                inport_ds,simInfo.indices.('temperature'),...
                T_amb, simInfo.names.ports.in.('temperature'));
            inport_ds = setElement(...
                inport_ds,simInfo.indices.pressure.('exit'),...
                p, simInfo.names.ports.in.pressure.('exit'));
            inport_ds = setElement(...
                inport_ds,simInfo.indices.pressure.('entry'),...
                p_amb, simInfo.names.ports.in.pressure.('entry'));
        else
            T_inletT = timeseries(T_inletT.Value,t,...
            'name','turbine_inlet_temperature');
            inport_ds = setElement(...
                inport_ds,simInfo.indices.('temperature'),...
                T_inletT, simInfo.names.ports.in.('temperature'));
            inport_ds = setElement(...
                inport_ds,simInfo.indices.pressure.('exit'),...
                p_amb, simInfo.names.ports.in.pressure.('exit'));
            inport_ds = setElement(...
                inport_ds,simInfo.indices.pressure.('entry'),...
                p, simInfo.names.ports.in.pressure.('entry'));   
        end
    case 'LYS'
        in_state.phi = phi;
        inport_ds = setElement(...
            inport_ds,simInfo.indices.('scale'),...
            scale, simInfo.names.ports.in.('scale'));
        if isCompressor
            in_state.T = T_amb;
            in_state.p = p_amb;
            inport_ds = setElement(...
                inport_ds,simInfo.indices.('in_state'),...
                in_state, simInfo.names.ports.in.('in_state'));
            inport_ds = setElement(...
                inport_ds,simInfo.indices.pressure.('back'),...
                p, simInfo.names.ports.in.pressure.('back'));
        else
            in_state.T = timeseries(T_inletT,t,...
            'name','turbine_inlet_temperature');
            in_state.p = p;
            inport_ds = setElement(...
                inport_ds,simInfo.indices.('in_state'),...
                in_state, simInfo.names.ports.in.('in_state'));
            inport_ds = setElement(...
                inport_ds,simInfo.indices.pressure.('back'),...
                p_amb, simInfo.names.ports.in.pressure.('back'));
        end
    otherwise
        error('No model author codenamed ''%s'' known.', P.Results.author);
end
%% Create Scenarios
%{
This part is hugely inspired from an automatically generated script. It can
be generated by opening a Simulink Model, then selecting 'Connect Input'
for an arbitrary Input Port. A Scenario dialogue opens, where data signals
for all the model's inputs can be created. One such set is called
'Scenario'. If multiple such sets are created, these scenarios can be run
one by one in a script. Matlab offers to 'Generate MATLAB Script' in this
window, generating a script that specifies all created scenarios to be run
in parallel and the output results to be saved. This is exactly what the
following code does, too. An array is created that contains multiple
'Datasets'. Each dataset contains *all* timeseries required to run the
model: one timeseries for each Input Port in the model. All these Datasets
are then summarized in the array. Afterwards, they are used as
'ExternalInput' for the model to be run. The models are run and the results
stored in another array. All Output ports record their data automatically,
no need for Displays, Signal Logging, Scopes etc. Create Array simIn of
certain length, then put a skeleton SimulationInput object for our chosen
model into each array cell:
%}
simIn(1:simInfo.number_of.('scenarios')) = ...
                Simulink.SimulationInput(simInfo.names.('model'));
%{
Set all simulation durations to the specified end time. This has the
advantage of not dirtying the model, since this change is only applied
temporarily and the model is reverted to its last saved state after this
simulation was run. Set MaxStep lower to force more precision in
variable-stepsize mode.
%}
simIn = setModelParameter(simIn, 'StopTime', num2str(t(end)));
simIn = setModelParameter(simIn, 'MaxStep', num2str(P.Results.max_step));
%{
For each Rotational Speed, create one Entry in the SimulationInput array.
First, create one timeseries object from the vector of rotational speeds.
Set its unit. Add this timeseries to the skeleton DataSet we pulled from
the model earlier. This completes it, since the other timeseries were
already loaded into the DataSet. So now, the DataSet can be put into the
ExternalInput slot for the SimulationInput object. Each SimulationInput is
different, so for-loop this (cannot be vectorized (?)). Also set the
Initial Guess for the Algebraic Constraint to different values, depending
on the rotational speed. This is important; if the first guess for p_1 is
'too wrong', an error occurs. The slower the speed, the closer p_1 is to
p_0. Refer to the local function itself for more info.
%}
switch P.Results.author
    case 'POV'
        for kScenario = 1:simInfo.number_of.('scenarios')
            n_ts = timeseries(n(kScenario),t,'name','Rotational_Speed');
            n_ts.DataInfo.Units = '1/s';
            inport_ds = setElement(...
                inport_ds,simInfo.indices.('rot_speed'),n_ts,...
                        simInfo.names.ports.in.('rot_speed'));
            simIn(kScenario).ExternalInput = inport_ds;
            simIn(kScenario) = setBlockParameter(simIn(kScenario), ...
                [simInfo.names.('model') '/Iteration'], 'InitialGuess',...
                IC_p1(n(kScenario), simInfo.('ambient_conditions').p));
            simIn(kScenario) = setBlockParameter(simIn(kScenario), ...
                [simInfo.names.('model') '/IC_Pi_D'], 'Value',...
                IC_Pi_D(n(kScenario)));
            simIn(kScenario) = setBlockParameter(simIn(kScenario), ...
                [simInfo.names.('model') '/IC_p_2'], 'Value', ...
                IC_p_2(n(kScenario), simInfo.('ambient_conditions').p));
        end
    case 'LYS'
        for kScenario = 1:simInfo.number_of.('scenarios')
            n_ts = timeseries(n(kScenario),t,'name','Rotational_Speed');
            n_ts.DataInfo.Units = '1/s';
            inport_ds = setElement(...
                inport_ds,simInfo.indices.('rot_speed'),n_ts,...
                        simInfo.names.ports.in.('rot_speed'));
            simIn(kScenario).ExternalInput = inport_ds;
        end
    otherwise
        error('No model author codenamed ''%s'' known.', P.Results.author);
end
%% Run Simulations
%{
Run using parsim, which runs simulations in parallel. The simulations it
runs are determined by the input cell array. Afterwards, add some Metadata
to Output. Also collect simulations that ran with errors. 'UseFastRestart'
cannot be used when setting block parameters with 'setBlockParameter'.
%}
simInfo.tau.start.('sim') = tic;
simOut = parsim(simIn,... `\label{line:teststand_parsim}`
        'ShowSimulationManager','on',...
        'RunInBackground','off',...
        'UseFastRestart','off');
%{
Iterate through the fields of the (sub-)struct that holds all the names for
the expected output ports. If not all the expected output ports are found
in the simulation output object, throw an error.
%}
for out_port = fieldnames(simInfo.names.ports.out)'
    assert(~isempty(...
    simOut.yout.find(...
    simInfo.names.ports.out.(out_port{:}))),...
    ['One or more of the expected output ports as defined in '...
    'the simInfo struct is/are missing. Please add it to the model.']);
end
%{
For good measure, make sure the relevant output ports have units assigned
to them. This avoids confusion and makes plotting more straightforward.
%}
assert(~isempty(simOut(1).yout.find(...
    simInfo.names.ports.out.('pressure_ratio')).Values.DataInfo.Units),...
            'Assign a Unit (1) for the Pressure Ratio Port.');
assert(~isempty(simOut(1).yout.find(...
    simInfo.names.ports.out.(...
    'enthalpy_stream')).Values.dotm.DataInfo.Units),...
            'Assign a Unit (kg/s) for the Mass Flux Port.');
assert(~isempty(simOut(1).yout.find(...
    simInfo.names.ports.out.('efficiency')).Values.DataInfo.Units),...
            'Assign a Unit (1) for the Efficiency Port.');
        
if exist('simOut','var')
    simOut = simOut.setUserString(['For Rot. Speed [krpm], ',...
            'see UserData.']);
    for kScenario = 1:simInfo.number_of.('scenarios')
         simOut(kScenario)=...
             simOut(kScenario).setUserData(n(kScenario)*60/1000);   
    end
end
simInfo.tau.stop.('sim') = toc(simInfo.tau.start.('sim'));

simInfo.errors.messages = {simOut.ErrorMessage};
simInfo.errors.indices = ~cellfun(@isempty, simInfo.errors.messages);

if any(simInfo.errors.indices)
    fprintf(['[%s] %u out of %u simulations ended erroneously. ',...
        'Failed rotational speeds (in krpm) were [index bracketed]:\n',...
        '[%s] %s.\n',...
        '[%s] These are removed for plotting.\n'],...
        datestr(now),...
        nnz(simInfo.errors.indices),...
        length(simOut),...
        datestr(now),...
        join(string(n(simInfo.errors.indices)*60/1000) + ...
        " [" +...
        string(find(simInfo.errors.indices)) +...
        "]", ", "),...
        datestr(now));
    n = n(~simInfo.errors.indices);
end
%% Prepare Results
%{
Cells have to be used, since the simulations probably take unequal amounts
of time-steps, and matrices need to be strictly rectangular, with equal
row/column widths. Cells may contain arbitrary content in each cell array
element. Preallocate Cells first. Assign Simulation values to cells (cells
can contain different datatypes, so no idea how to vectorize this). Grab
the maximum choke flux for each scenario; currently, these values are
absolutely constant over time (maybe in the future, they aren't anymore).
In any case, condense the values into a single one by building their
mean(). Also remove the very first entry, since it is probably ruined from
auxiliary initial conditions (like zero mass flux). If it is not but is
still deleted, a single value won't be missed. So delete anyway. The 'deal'
functions doesn't seem to work for dealing an empty '[]' entry. Afterwards,
remove all empty entries in the created cells (for example, if a simulation
did not run at all due to errors [losses too high and speed too low;
Algebraic Constraint did not converge]).
%}
[Pi.complete, dotm.complete, eta.complete, torque.complete,...
    T_after.complete] = ...
    deal(cell(1,simInfo.number_of.('scenarios')));

for kScenario = 1:simInfo.number_of.('scenarios')
    dotm.complete{kScenario} = ...
        simOut(kScenario).yout.find(...
        simInfo.names.ports.out.('enthalpy_stream')).Values.dotm.Data;
    Pi.complete{kScenario} = ...
        simOut(kScenario).yout.find(...
        simInfo.names.ports.out.('pressure_ratio')).Values.Data;
    eta.complete{kScenario} = ...
        simOut(kScenario).yout.find(...
        simInfo.names.ports.out.('efficiency')).Values.Data;
    torque.complete{kScenario} = ...
        simOut(kScenario).yout.find(...
        simInfo.names.ports.out.('torque')).Values.Data;
    T_after.complete{kScenario} = ...
        simOut(kScenario).yout.find(...
        simInfo.names.ports.out.('enthalpy_stream')).Values.T.Data;
end
%{
Truncate the result cell arrays slightly. At least cut off first and last
entry. Otherwise, we run into off-by-one errors. '' 'un',0 '' is shorthand
for '' 'UniformOutput', false ''. Cut of a certain percentage of initial
junk values.
%}
cutoff = 0.02;
truncvec = @(vec) vec(ceil(length(vec)*cutoff):...
                        end-ceil(length(vec)*cutoff));
Pi.oboe = cellfun(truncvec, Pi.complete, 'un', 0);
dotm.oboe = cellfun(truncvec, dotm.complete, 'un', 0);
eta.oboe = cellfun(truncvec, eta.complete, 'un', 0);
torque.oboe = cellfun(truncvec, torque.complete, 'un', 0);
T_after.oboe = cellfun(truncvec, T_after.complete, 'un', 0);
%% Look-Up Table Generation
%{
These quantities are corrected/reduced, so that they are rendered
independent of the environmental conditions they were created with. The
total conditions at inlet are the same as static conditions since there is
no initial velocity in front of the machine in our model. The LUT are saved
in their own Data Dictionary, which needs to be loaded. Its name was
already defined earlier.
%}
n_red = n / sqrt(in_state.T.Data(1));
%{
Turbine and Compressor are normalized differently. The reduced speed is the
same for all. The expression for the mass flow is still dotm_red =
dotm*sqrt(T)/p. T and p are constant in the compressor test, but p
increases linearly in the turbine case. At each point, the normalization
needs to be adjusted. Derive new 'p' cell array from existing dotm array,
so that they are the same size. Initialize as zero to notice errors more
clearly. Then resample existing pressure timeseries to be same length as
each cell array vector. Then divide element-wise.
%}
if isCompressor
    dotm.red = cellfun(@(x)...
        (x * sqrt(in_state.T.Data(1))) / ...
        (in_state.p.Data(1)), dotm.oboe, ...
        'UniformOutput', false);
else
    p_rsmpl = cellfun(@(x) zeros(size(x)), dotm.oboe, 'un', 0);
    p_rsmpl = cellfun(@(x) linspace(p.Data(1), p.Data(end), length(x))',...
        p_rsmpl, 'un', 0);
    dotm.red = cellfun(@(x, y) x./y*sqrt(in_state.T.Data(1)),...
            dotm.oboe, p_rsmpl,'un', 0);
end

dotm_red = create_turbo_LUT(dotm.red, Pi.oboe, n_red);    
    dotm_red.Table.Min = 0;
    dotm_red.Table.Unit = '(kg*K^0.5)/(s*Pa)';
    dotm_red.Table.Description = ['All reduced mass flows ',...
        'computed in simulation. ',...
        'Rows are (reduced) rot. speeds; columns are (static) ',...
        'pressure ratios.'];
    dotm_red.StructTypeInfo.Name = 'dotm_red';
    dotm_red.Breakpoints(1).Unit = '1/(s*K^(0.5))';

eta_tot = create_turbo_LUT(eta.oboe, Pi.oboe, n_red);
    eta_tot.Table.Min = 0;
    eta_tot.Table.Max = 1;
    eta_tot.Table.Unit = '1';
    eta_tot.Table.Description = 'Total-to-Total Efficiency';
    eta_tot.StructTypeInfo.Name = 'eta';
    eta_tot.Breakpoints(1).Unit = '1/(s*K^(0.5))';

M = create_turbo_LUT(torque.oboe, Pi.oboe, n_red);
    M.Table.Min = 0;
    M.Table.Unit = 'N*m';
    M.Table.Description = 'Torque';
    M.StructTypeInfo.Name = 'M';
    M.Breakpoints(1).Unit = '1/(s*K^(0.5))';

T_after = create_turbo_LUT(T_after.oboe, Pi.oboe, n_red);    
    T_after.Table.Min = 0;
    T_after.Table.Unit = 'K';
    T_after.Table.Description = 'Exit Temperature';
    T_after.StructTypeInfo.Name = 'T_after';           
    T_after.Breakpoints(1).Unit = '1/(s*K^(0.5))';

cnamesc = [simInfo.names.turbocharger, '_', ...
    upper(simInfo.names.('machine_type')(1)),...
    '_',num2str(P.Results.scale*100)];
%{
We could put all LUT objects into one Struct whose fieldname is the
turbocharger's name, but Simulink is unable to read LUT objects out of
structs.
%}
add_to_DD(simInfo.names.('dd_maps'), [cnamesc,'_dotm_red'], dotm_red);
add_to_DD(simInfo.names.('dd_maps'), [cnamesc,'_eta_tot'], eta_tot);
add_to_DD(simInfo.names.('dd_maps'), [cnamesc,'_M'], M);
add_to_DD(simInfo.names.('dd_maps'), [cnamesc,'_T'], T_after);
%% Plot in Matlab
if P.Results.plot
%{
Construct labels from the port names we find in the simulation. For this to
work as in the original make sure to set groot interpreter to latex.
%}
if P.Results.redplot
    label.('dotm') = ['\boldmath\bfseries ',simOut(1).yout.find(...
        simInfo.names.ports.out.('enthalpy_stream')...
                                ).Values.dotm.Name, '\_red ['...
        simOut(1).yout.find(...
        simInfo.names.ports.out.('enthalpy_stream')...
                            ).Values.dotm.DataInfo.Units.Name,...
                            ' * K\^ 0.5/Pa', ']'];
else
    label.('dotm') = ['\boldmath\bfseries ',simOut(1).yout.find(...
        simInfo.names.ports.out.('enthalpy_stream')...
                                ).Values.dotm.Name, ' ['...
        simOut(1).yout.find(...
        simInfo.names.ports.out.('enthalpy_stream')...
                            ).Values.dotm.DataInfo.Units.Name,']'];
end

label.('Pi') = ['\bfseries $\',simOut(1).yout.find(...
    simInfo.names.ports.out.('pressure_ratio')...
                            ).Name, '$ ['...
    simOut(1).yout.find(...
    simInfo.names.ports.out.('pressure_ratio')...
                        ).Values.DataInfo.Units.Name,']'];

label.('efficiency') = ['\bfseries $\',simOut(1).yout.find(...
    simInfo.names.ports.out.('efficiency')...
                            ).Name, '$ ['...
    simOut(1).yout.find(...
    simInfo.names.ports.out.('efficiency')...
                        ).Values.DataInfo.Units.Name,']'];
%{
Prepare Figure for the following Plot. Set axes limits such that the
mass flow axis starts at 0, pressure ratio axis at 1. Upper limit 'inf'
lets Matlab decide automatically.
%}
plots = figure;
plots.Name = ['Turbomachine Performance Map at ', ...
    num2str(P.Results.scale*100), ' %'];

if isCompressor
    subplot(2,1,1)
        if P.Results.redplot
            turboplot(dotm.red, Pi.oboe);
        else
            turboplot(dotm.oboe, Pi.oboe);
        end
        ylim([1 5.5]);
        title(['\bfseries ', simInfo.names.('machine_type'),...
            ' Performance Map for ',...
            simInfo.names.('turbocharger')]);
        xlabel(label.('dotm'));
        ylabel(label.('Pi'));
    subplot(2,1,2)
        if P.Results.redplot
            turboplot(dotm.red, eta.oboe);
        else           
            turboplot(dotm.oboe, eta.oboe);
        end
        ylim([0 1]);
        title(['\bfseries ', simInfo.names.('machine_type'),...
            ' Efficiency Map for ',...
            simInfo.names.('turbocharger')]);
        xlabel(label.('dotm'));
        ylabel(label.('efficiency'));
else
    subplot(2,1,1)
        if P.Results.redplot
            turboplot(Pi.oboe, dotm.red);
        else
            turboplot(Pi.oboe, dotm.oboe);
        end
        ylim([0 inf]);
        title(['\bfseries ', simInfo.names.('machine_type'),...
            ' Performance Map for ',...
            simInfo.names.('turbocharger')]);
        xlabel(label.('Pi'));
        ylabel(label.('dotm'));
    subplot(2,1,2)
        turboplot(Pi.oboe, eta.oboe);
        ylim([0 1]);
        title(['\bfseries ', simInfo.names.('machine_type'),...
            ' Efficiency Map for ',...
            simInfo.names.('turbocharger')]);
        xlabel(label.('Pi'));
        ylabel(label.('efficiency'));
end
subplot(2,1,1)
if P.Results.redplot
    lgd = legend( cellfun(...
            @num2str, num2cell(n_red), 'UniformOutput', false) ,...
        'Location', 'NorthWest');
    title(lgd, 'n\_red [1/(s*K\^ 0.5)]');    
else
    lgd = legend( cellfun(...
            @num2str, num2cell(n*60/1000), 'UniformOutput', false) ,...
        'Location', 'NorthWest');
    title(lgd, 'n [krpm]');
end
end
%% Write to File
%{
Intended to write to CSV for plotting in LaTeX. Create a file for each run;
awful approach, but tables/matrices are not allowed to be of unequal
size. Trouble is alleviated using foreach in pgfplots.
%}
if P.Results.write
    tbl = cellfun(@(x, y, z) [x y z], Pi.oboe, dotm.red, eta.oboe, 'un',0);
    for kScenario = 1:simInfo.number_of.('scenarios')
        tbldata = tbl{kScenario};
        fl.name = [num2str(P.Results.scale*100), '_', ...
            num2str(round(n_red(kScenario))), '.csv'];
        fl.path = fullfile('X:\', 'Education', 'University', '13_MT', ...
            'LaTeX', 'data', 'maps', simInfo.names.('turbocharger'),...
            upper(simInfo.names.('machine_type')(1)), fl.name);
        csvwrite(fl.path, tbldata);
    end
end
%% Finish
simInfo.tau.stop.('script') = toc(simInfo.tau.start.('script'));
fprintf(['[%s] Script finished. ',...
    'It took %d minute(s) and %2.1f seconds, %d minute(s) ',...
    'and %2.1f seconds of which for the simulation of the ',...
    '%d scenario(s).\n'],...
    datestr(now), floor(simInfo.tau.stop.('script')/60),...
    rem(simInfo.tau.stop.('script'),60),...
    floor(simInfo.tau.stop.('sim')/60),...
    rem(simInfo.tau.stop.('sim'),60), simInfo.number_of.('scenarios'));
end
%% Local Functions (i.e. not nested)
function idx = idxSearch(port_name,dataset)
%{ 
Find Index of a port with a specific name in the model. The model is
represented by its dataset. The input for 'port_name' is literally the name
of the input port in the model that this is trying to match. If Strings
compare succesfully, index is found. Thus, Break out of loop. However, if
search comes to its last index and the correct string is still not found
there, return fatal error: input not present.
%}
    for idx = 1:dataset.numElements
        if strcmp(dataset{idx}.Name,port_name)
            break
        end
        if idx == dataset.numElements...
            && ~strcmp(dataset{idx}.Name,port_name)
            error(['Could not find an Input Port named ''%s''. ',...
                'Check that it is spelled correctly, ',...
                'and that there are no trailing spaces or linebreaks.'],...
                port_name);
        end
    end
end

function map = turboplot(x,y)
%{
Providing some simple but common plot commands. Create 'map' struct to hold
the lines data. Modify all occurrences by deal()ing to them.
%}
    grid on
    hold on
    map = cellfun(@plot, x, y);
    [map.Marker] = deal('o');
    [map.LineWidth] = deal(1);
    [map.MarkerSize] = deal(3);
end

function InitialCond_p1 = IC_p1(n, p_0)
%{
This function yields an initial condition for the pressure p_1. This
pressure is part of the Algebraic Loop. p_1 is put out by the Algebraic
Constraint (the iterator). Depending on the rotational speed the machine is
run at, the pressure at inlet p_1 varies strongly. If the machine rotates
very slowly, little energy is introduced; p_1 is very close to p_0 then.
The faster the rotational speed (as always, in RPS), the stronger the first
outflow is and the lower p_1 is. From manual experiments, the function
p_1(n, t=0) -- meaning the initial condition for p_1 depending on rot.
speed -- was found to resemble the sigmoid curve below, represented using
tanh. Careful: all parameters are determinde manually. They need to be
redone if a new machine does not fit anymore. Simulink is quite sensitive
to a correct Initial Guess for the Algebraic Constraint. However, if the
general direction is okay, we good. Probably, a simple linear curve fit
would do a job just as good; it wouldn't be as good in the extreme
rot-speed regions. Lastly, convert numeric output to string, because
Simulink expects strings for 'setBlockParameter'.
%}
    InitialCond_p1 = (1 - tanh(n/600 - 3) ) * p_0/2;
    InitialCond_p1 = num2str(InitialCond_p1);
end

function InitialCond_p_2 = IC_p_2(n, p_0)
%{
After introducing the diffuser, a new pressure p_3 appeared. Pressure p_2
(behind impeller and in front of diffuser) is now part of the loop and
requires a special initial condition. The relation was found out by hand
and probably needs tweaking for new machines. The initial values are set to
be a portion of the inlet (ambient) pressure.
%}
    InitialCond_p_2 = (1 - 0.1337 * n / 666 )*p_0;
    InitialCond_p_2 = num2str(InitialCond_p_2);
end

function InitialCond_Pi_D = IC_Pi_D(n)
%{
The Initial Condition for the initial pressure ratio provided by the
diffuser is also rather sensitive to its Initial Condition
%}
    InitialCond_Pi_D = n / 2250 + 0.8;
    InitialCond_Pi_D = num2str(InitialCond_Pi_D);
end

function LUTobj = create_turbo_LUT(Your_c_arr, Pi_c_arr, n)
%{
Returns finished 2D-LookupTable Object from Cell Arrays of Pi (used for
Breakpoint-Columns) and the desired quantity (used for table data). The
latter can be dotm, eta etc. Also requires vector of rotational speeds
(used for Breakpoint-Rows). The pressure ratios resulting from simulation
might differ wildly. Due to variable step sizes, all computed
timeseries/vectors are almost guaranteed to be of different lengths.
Additionally, the contained values are going to have a large amount of
significant digits (especially for type double). These cannot be arranged
on the same grid for these reasons; rounding required. The rounding takes
place until a high enough resolution to round even the most fine-grained
results is reached. Once all results are rounded, they can be arranged on
the same grid together. This is the prerequisite for breakpoints for lookup
tables.
%}
decimalres = 0;
repetitions_occurring = true;

while repetitions_occurring
    decimalres = decimalres + 1;
    
    res = 10^(-decimalres);
    
    rounded = cellfun(@(x) round(x, decimalres), Pi_c_arr,...
                            'UniformOutput', false);
                        
    check_reps = @(arr) length(arr) ~= length(unique(arr));
                        
    repetitions = cellfun(check_reps, rounded);
    
    repetitions_occurring = any(repetitions);
    
    if decimalres > 6
        error(['In rounding, a very high step resolution of %d ',...
            'was reached. This is probably an issue originating from ',...
            'elsewhere.'], res)
    end
end
%{
This is weird numerical glitch, but if this is not put trough round()
again, we run into floating point limitations and the check for intersect()
later on might fail. This makes sense, since now both elements are put
through the round() function. If only one of them is, originally equal
entries might differ in one Bit (this can be checked via hexadecimal
representation, num2hex()). Generate length of Pi breakpoints from the
longest vector / largest Pi found in all arrays in the cell array.
%}
Pi_min = min(cellfun(@min, Pi_c_arr));
Pi_max = max(cellfun(@max, Pi_c_arr));
%{
The values found for Pi_min and Pi_max might not fit into the scheme
governed by the resolution. If they are uneven values, they might get
rounded down or up by forcing the resolution on them, and this might give
unexpected results. Therefore, do this manually to be certain. Round
max. value UP (ceiling), min. value DOWN (floor it).
%}
Pi_max = ceil(Pi_max / res) * res;
Pi_min = floor(Pi_min / res) * res;

Pi_bpdata = round(Pi_min:res:Pi_max, decimalres);
%{
Use NaN for Initialization this time in order to use fillmissing(), which
does not register Zeros or Ones as 'missing'.
%}
LUTdata = NaN(length(n), length(Pi_bpdata));
%{
Loop over Scenarios (still not able to vectorize (?)). Intersect() finds
all values the two inputs have in common and then outputs them as well as
their indices in both arrays. Create a new table with data for the input
quantity's cell array that is as large as the breakpoint vector. At the new
indices, insert the computed values for it. This will likely leave NaN
entries inbetween. These are interpolated across later on.
%}
for kScenario = 1:length(n)
    [~, ind.a, ind.b] = intersect(Pi_bpdata, rounded{kScenario});
    assert(length(ind.b) == length(rounded{kScenario}),...
        ['The rounded pressure ratios of scenario %d are not ',...
        'fully included in the breakpoint set for Pi. ',...
        'This can happen if floating point limitations occur, so that',...
        ' two equal values are off by one bit.'], kScenario)
    LUTdata(kScenario, ind.a) = Your_c_arr{kScenario};
end
%{
Interpolate; using linear seems too simple, since the functions are always
curved and *strictly* non-linear. Specifying '2' makes fillmissing() work
row-by-row, across columns. 'none' for 'EndValues' leaves all trailing or
initial entries (which could only be extrapolated, which would be
*absolutely* wrong) at their original. This step is very important. It
could leave NaN at the beginning of the arrays, which are cropped off.
%}
LUTdata = fillmissing(LUTdata, 'spline', 2, 'EndValues', 'none');

LUTdata = truncarr(LUTdata, 1, 0);
Pi_bpdata = truncarr(Pi_bpdata, 1, 0);

    Pi_bp = Simulink.Breakpoint;
        Pi_bp.Breakpoints.Value = Pi_bpdata;
        Pi_bp.Breakpoints.Unit = '1';
        Pi_bp.Breakpoints.Min = 1;
        Pi_bp.Breakpoints.Max = Pi_bpdata(end);
        Pi_bp.Breakpoints.Description = ['Range of (static!) pressure ',...
            'ratios tested in simulation.'];
        Pi_bp.Breakpoints.FieldName = 'Pi';
        Pi_bp.StructTypeInfo.Name = 'Pi';
        
    n_bp = Simulink.Breakpoint;
        n_bp.Breakpoints.Value = n;
        n_bp.Breakpoints.Min = n(1);
        n_bp.Breakpoints.Max = n(end);
        n_bp.Breakpoints.Description = 'All rot. speeds tested.'; 
        n_bp.Breakpoints.FieldName = 'n';
        n_bp.StructTypeInfo.Name = 'n';
    
    LUTobj = Simulink.LookupTable;
        LUTobj.Table.Value = LUTdata;
        LUTobj.Breakpoints(1) = n_bp.Breakpoints;
        LUTobj.Breakpoints(2) = Pi_bp.Breakpoints;
end

function add_to_DD(name_dd, name_entry, Entry)
% Takes in a Data Dictionary's name and the name of the Entry intended for
% it. Also takes in the object to be created in the entry. This could be a
% LUT Object.
dObj = Simulink.data.dictionary.open(which(name_dd));
dSectObj = getSection(dObj,'Design Data');
try
    addEntry(dSectObj, name_entry, Entry);
catch ME
    switch ME.identifier
        case 'SLDD:sldd:EntryAlreadyExists'
            warning(['An entry ''%s'' already existed in the ',...
                'Data Dictionary ''%s''. It was not reassigned. ',...
                'Delete it and run this function again. ',...
                'Check its creation time property for reference.'],...
                name_entry, name_dd);
        otherwise
            rethrow(ME)
    end
end
end
\end{lstlisting}

\end{appendices}

%To check whether fonts work correctly:
%This is the main font. \textbf{Bold main font}. \textit{Italic main font} \textbf{\textit{Bold Italics Main Font}} \textsc{Roman Small Capitals}\\
%
%\ttfamily Hello this is Consolas!! \textit{Consolas Italic} \textbf{Consolas Bold} \textbf{\textit{Consolas Bold Italic}} \textsc{TT Small Capitals}\\
%
%\sffamily Hello this is Sans!! \textit{Sans Italic} \textbf{Sans Bold} \textbf{\textit{Sans Bold Italic}}\textsc{Sans Small Capitals}
%
%\begin{equation}
%\sum 1 + 2 = 45^2, \iiiint \SI{45}{\meter}
%\end{equation}

%\SI{45}{\meter}